%%% Time-stamp: <2024-06-01 14:49:35 vladimir>
%%% Copyright (C) 2019-, 2024, 20242024  Vladimir G. Ivanović
%%% Author: Vladimir G. Ivanović <vladimir@acm.org>
%%% ORCID: https://orcid.org/0000-0002-7802-7970

\chapter{Colophon}\indent\small\OnehalfSpacing%

This dissertation was created entirely with free, open source software (FOSS), except for a single proprietary program used to convert PDFs into spreadsheets. The typefaces, editor, markup language, reference manager, operating system, and all utilities are FOSS. 
\begin{center}
  \textbf{\adforn{21}\qquad\adforn{21}\qquad\adforn{11}\qquad\adforn{49}\qquad\adforn{49}}
\end{center}
The body and headings were set in 12pt Alegreya. The Alegreya family of serif \& sans serif typefaces was designed by Juan Pablo del Peral of Huerta Tipográfica in 2011 and immediately won praise and awards. It is a classic Renaissance typeface, a kind that was first developed in the fourteenth and fifteenth centuries in northern Italy. It comes in Regular, Medium, Bold and Black weights, all of which are available in Roman and Italic styles. There is a full set of Greek and Cyrillic letters as well as Latin small caps. All have a full set of ligatures, and Old Style and Lining numerals. Notably, all the numerals share the same width so they line up regardless of which style is being used. (Multiplication using Roman numerals, anyone?) If any criticism can be leveled against the Alegreya superfamily, it is that the family does not include display sizes and does not contain swash characters. Otherwise, it is nearly perfect.
\begin{center}
  \textbf{\adforn{21}\quad\quad\adforn{21}\quad\quad\adforn{11}\quad\quad\adforn{49}\quad\quad\adforn{49}}
\end{center}
The programs \TeX{} \& \LaTeX{}  and the document class \texttt{memoir} were used to format this dissertation. \LaTeX  was created by Leslie Lamport as a user-friendly version of one of the first digital typesetting systems, \TeX . \TeX  is one of the masterpieces of computer programming whose author, Donald Knuth, won the Turing Award in 1974. It is a testament to Knuth's brilliance as both a mathematician and as a programmer that \TeX  is still in use more than four decades later, and arguably has no peers when it comes to typesetting complex mathematics and scientific material. It is, however, awkward to use and hard to learn. Fortunately, Leslie Lamport wrapped \TeX  in a macro system, \LaTeX , which was orders of magnitude easier to use than \TeX  itself.

LaTeX is extraordinarily flexible as shown by the thousands of packages which implement specialized tasks. Currently, CTAN (the Comprehensive TeX Archive Network) has just shy of 6000 packages which can be downloaded. One of those packages implements the class \texttt{memoir} that was used here. It was written by Peter Wilson, and released in 2001. (I'm listed as a contributor to \texttt{memoir}, but in truth I really only just corrected some minor typos.)
\newpage
\begin{center}
  \textbf{\adforn{21}\quad\quad\adforn{21}\quad\quad\adforn{11}\quad\quad\adforn{49}\quad\quad\adforn{49}}
\end{center}
Wilson's muse is Robert Bringhurst, author of \textit{The Elements of Typographic Style}, which some consider to be the definitive book on typography and book design. It is certainy the most elegant. The package \texttt{memoir} would undoubtedly meet with Bringhurst's approval.  The class \texttt{memoir} provides in one package nearly everything a person needs to produce what Knuth calls ``beautiful books''.
\begin{center}
  \textbf{\adforn{21}\quad\quad\adforn{21}\quad\quad\adforn{11}\quad\quad\adforn{49}\quad\quad\adforn{49}}
\end{center}
One particularly important FOSS program in academia is Zotero. It manages and maintains a bibliographic database and provides citations on demand. It, along with the text editor GNU Emacs (``an operating system disguised as an editor'') and the package \texttt{reftex}, cooperate with the class \texttt{memoir} to provide a complete system for writing scholarly papers, theses, reports, and dissertations.
\begin{center}
  \textbf{\adforn{21}\quad\quad\adforn{21}\quad\quad\adforn{11}\quad\quad\adforn{49}\quad\quad\adforn{49}}
\end{center}
All of these program run on Arch Linux, a particular distribution of GNU Linux, itself a version of Unix. It is notable that GNU Linux, GNU Emacs, \TeX{} and \LaTeX{} are all programs that originated decades ago, are still actively used, have never been truly replaced or superseded, and are constantly being improved. They share a common set of characteristics: their fundamental architecture is sound, extensibility is a core feature, and they and thousands of specialized packages are freely available. I predict that iPhones (which run a version of Unix) will be the faintest of memories when GNU Unix, GNU Emacs, and \TeX{} and \LaTeX{} start to fade from view.

%%% Local Variables:
%%% mode: latex
%%% TeX-master: "Rocketship_Education-An_Exploratory_Public_Policy_Case_Study"
%%% End:
