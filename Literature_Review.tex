%%% Time-stamp: <2024-02-27 13:39:21 vladimir>
%%% Copyright (C) 2019-2024 Vladimir G. Ivanović
%%% Author: Vladimir G. Ivanović <vladimir@acm.org>
%%% ORCID: https://orcid.org/0000-0002-7802-7970
\chapter{A Review of the Literature}\label{ch:litreview}\indent%
Rocketship is part of a context that has a long history. This chapter reviews that context; what other researchers and scholars have said about the origins of charter schools, their history, and their ostensible goals before characterizing the finances of public schools in California and then the unique aspects of charter school finance. 

\index{public education!American!alleged failure of|(}
American public education has – allegedly – been a failure, at least ``[a]ccording to highly publicized NAEP results in the mid 1980s'' \parencite{Gove.Meier2000}. \textcite{Berliner.Glass2014} in \citetitle{Berliner.Glass2014} refute those myths which have been advanced to show that American's schools are in a crisis, and hence, in desperate need of reform. It turns out, this urge for reform has a long history: America's schools have judged as needing reform ever since the idea of free public education took hold in the early 1800s.\footnote{Wikipedia has an excellent summary article on \textit{Education in the United States} available at \url{https://en.wikipedia.org/wiki/Education_in_the_United_States}.} Since then, a succession of educators and reports have documented the abysmal [sic] state of American education.
\index{public education!American!alleged failure of|)}

\section{The Birth of American Public Education}\label{sec:birth-amer-publ}\indent%

Prior to the Civil War, Horace Mann introduced widely copied reforms \parencite%[147]
{Pulliam.VanPatten2007} into the existing system of education which was then not free, not open to all, and not compulsory. Those schools had hardly changed since the founding of the Boston Latin School on April 23, 1635. In the early 1900s, John Dewey, an educational leader of the Progressive Era (1896–1916) preached reform, but it was not until the publication of \citetitle{Gardner1983} in 1983 that the modern zeal for education reform took form. \citetitle{Gardner1983} was the most influential of roughly 30  major education reform reports listed by \textcite%[252]
{Pulliam.VanPatten2007} starting in 1982 and continuing up until 2005.

That American public education needed reform was repeated constantly, mainly by conservatives, despite underwhelming evidence of its veracity and substantial evidence to the contrary. Through constant repetition, the need for reform has become accepted wisdom. The answer to this need was to take the government's ``monopoly in education'' (Milton Friedman's characterization) out of the hands of faceless bureaucrats and subject it to the rigors of free markets which would, it was asserted with scant evidence and with the complete absence of a theory of action, increase efficiency, choice, and quality. Thus vouchers and charter schools were legitimized.

No amount of research, it seems, can dispel the \textit{idée fixe} that American education is in dire straits, and further, piecemeal changes were simply not enough to make substantive changes. No matter what \textcite{Henig1994} or \textcite{Berliner.Biddle1997} or \textcite{Nichols.etal2007} or \textcite{Glass2008} or \textcite{Berliner.Glass2014} wrote, the idea that American education needed fundamental, pervasive reform persisted; education reform was an evidence-free endeavor.

\citeauthor{Garcia2018} writes in \citetitle{Garcia2018}
\begin{quotation}\noindent
The four primary arguments put forth in support of school choice are the elimination of government bureaucracies, the interjection of competition into education through market forces, the promotion of parental choice as the most granular form of local control, and school choice as the ``new'' civil rights issue of our time.\footnote{Lest \citeauthor{Garcia2018} be tarred as anti-school choice, he is merely following Anatol Rapoport's Rules for Constructive Criticism, the first of which is to restate the argument of the person you are criticizing better than they themselves have done.} \parencite[55]{Garcia2018}.
\end{quotation}
What is noteworthy is that none of the four arguments are about student achievement or attainment. A poorly staffed, badly run, charter school located in a dangerous neighborhood is as capable of satisfying the four requirements as is a high quality charter school. Whatever school choice is about, it iss not about students and how well they are doing.

To be clear, it is not the case that every American school is a model for the rest of the world: systematic, persistent, pervasive inequities and injustices abound and have been powerfully written about in \textcite{Kozol1992} and again in \textcite{Kozol2005}, \textcite{Valenzuela1999}, \textcite{Heitzeg2009}, and \textcite{Roithmayr2021}. The Coleman Report \parencite{Coleman1966} concluded that ten years after \textit{Brown v. Board of Education}, American schools were still segregated and were still unequal. Surprisingly and contrary to the expectations of many, the report laid most of the blame for unequal educational outcomes on systematic, persistent, pervasive inequalities and injustices outside of schools. The report said,
\begin{quotation}\noindent
Taking all these results together, one implication stands out above all: That schools bring little influence to bear on a child's achievement that is independent of his background and general social context; and that this very lack of an independent effect means that the inequalities imposed on children by their home, neighborhood, and peer environment are carried along to become the inequalities with which they confront adult life at the end of school. \parencite[325]{Coleman1966}.
\end{quotation}
The report concluded that family background, the socioeconomic background of a school, and a student's sense that they were in control of their lives were more important than race-based disparities in explaining the black-white achievement gap \parencite{Pearce2016}.

 \textcite{Downey2020}, using two ECLS-K studies, 1998 and 2011, supports this conclusion but in a slightly different way. He finds that academic inequality is reduced when children are in school, and increases when children are not in school, i.e.~during the summer, which runs counter to the notion that schools exacerbate the achievement gap.

None of this should be a surprise because it is also clear that those schools have been systematically underfunded for decades; their dismal performance is more likely the result of the poverty of their neighborhoods and their lack of funding than it is the other way around. For example, the California School Boards Association's (CSBA) Education Legal Alliance Adequacy Committee found that there exists a ``substantial gap in funding between what K-12 education [in California] receives and what K-12 education needs even to meet the standards prescribed by the state \parencite[\textit{iii}]{Bray2015}. \textcite{Baker.etal2018} in their aptly titled report \citetitle{Baker.etal2018}, develop a \textit{National Education Cost Model} \parencite%[5]
{Baker.etal2018} which accounts for regional cost differences as well different funding levels to show that inadequate funding is present throughout the United States. \textcite{Garcia2018} says in \citetitle{Garcia2018} that the ``existence and importance of the issues that reformers believe plague public education are based as much on tradition and reputation as they are on tangible research evidence'' \parencite[54]{Garcia2018}. Finally, and tellingly, grossly inadequate funding is a characteristic of communities that are racially segregated and which are not white \parencite{Darling-Hammond2012, Rothstein2017}.

\textcite{Henig1994}'s book, \citetitle{Henig1994}, which came out a mere three years after the passage of the nation's first state charter school law in Minnesota\footnote{Laws of Minnesota 1991, chapter 265, article 9, section 3} and two years after the second in California\footnote{Education Code, Title 2, Division 4 Part 26.8, §47600 \textit{et.\ seq}} lays out a key argument against charter schools. Henig says, ``[T]he real danger in the market-based choice proposals is not that they might allow some students to attend privately run schools at public expense, but that \emph{they will erode the public forums in which decisions with societal consequences can democratically be resolved}.'' (emphasis added) \parencite[\emph{xiii}]{Henig1994}. Translated, this means that the decisions about public education's form and content are not going to be made by parents and teachers, but by people who do not have a stake in the outcome. It is now a matter of badly misaligned incentives. %chktex 38

But even before that, in 1982, Earl Craig, Jr.\ attached a minority report to \citetitle{CitizensLeagueEducationAlternativesCommittee1982} which advocated for vouchers. He says in a paragraph that is as accurate today, forty years later, as it was in 1982: 

\begin{quotation}\noindent
\noindent{}In conclusion, this report is part of a national movement toward privatization of public services and responsibilities. I believe this movement will have the eventual result of a complete retreat by this society from a societal responsibility for the powerless who are difficult or expensive to educate, house, protect, etc. I believe the committee and board majority when they say that they are committed to equal access and equity. They say, trust that we will do the right thing. I do trust them, I do not trust the societal momentum of which vouchers is a part. It is a very destructive wave that has caught up many good people. It scares me to death. \parencite[48]{CitizensLeagueEducationAlternativesCommittee1982}
\end{quotation}

The belief that that American schools were in crisis due to poor academic outcomes, sclerotic teachers resistant to change, ineffective and bureaucratic administrators more concerned with job safety than educating children is simply not supported by the evidence. But the idea that American schools are in crisis has been continuously promoted, and sheer repetition has turned fiction turned into fact, and this ``manufactured crisis'', to use David Berliner and Bruce Biddle's turn of phrase \parencite{Berliner.Biddle1997}, has been used to justify school choice in the form of vouchers and charter schools. But charter schools did not actually take off until ``education reformers across party lines realized that charter school laws could be crafted in ways that made it possible to open nonunion public schools, or even allow public schools to be managed by for-profit companies.'' \parencite[172]{Goldstein2015}

This literature review will first examine charter schools, their origins and the early research, before reviewing the types of charters which exist. It then examines the various models of charter schools such as virtual charter schools, charters which use blended learning, and charter management organizations before taking a closer look charter schools in Santa Clara County and in Rocketship in particular. It ends with a consideration of the finances and financing of charter schools.

\section{A History of Charter Schools}\label{sec:cs-history}\indent

Charter schools (privately run, but publicly financed schools) have an ugly racist origin in the post-\textit{Brown v Board of Education} era as a method of evading the U.S. Supreme Court's mandate to educate both black and white Americans equally and not separately. Fifty years later, charter schools turned segregation academies into the preferred vehicle for privatizing public schools for profit while maintaining segregation.

\subsection{The Origins of Charter Schools in Segregation}\label{sec:origins}\indent

The first charter schools were not founded for educational or economic reasons. Charter schools had their origin in the aftermath of \textit{\citetitle{Warren1954}}\index{Brown v. Board of Education}. ``[\textit{Brown}] was the genesis of school choice as a public policy mechanism.'' \parencite[8]{Garcia2018} In the Deep South, academies sprung up as part of the massive resistance to the U.S. Supreme Court's unanimous 1954 ruling which answered the question,
\begin{quote}
Does segregation of children in public schools solely on the basis of race, even though the physical facilities and other `tangible' factors may be equal, deprive children of the minority group of equal educational opportunities? \parencite[9]{Warren1954}
\end{quote}

\noindent{} with ``We believe that it does.'' (ibid) %chktex 38

In order to circumvent \textit{Brown\index{Brown v. Board of Education}}, white parents in eleven states formed thousands of private schools, and until the early 1970's, these segregation academies received public funds \parencite%[81]
  {Rooks2017}. These origins of  charter schools have been amply documented, in \textcite{Frankenberg.etal2010}, \textcite{Frankenberg.etal2011}, and especially in \textcite{Suitts2019} and \textcite{Suitts2020}. \citeauthor{Alexander2011} in \citetitle{Alexander2011} quotes \textcite[52]{Rosenberg1991} ``The statistics from the Southern states are truly amazing. For ten years, 1954–1964, virtually \textit{nothing happened}.'' [emphasis in \parencite[223]{Alexander2011}]. She goes on to say, %chktex 38
\begin{quotation}\noindent
Not a single black child attended an integrated public grade school in South Carolina, Alabama, or Mississippi as of the 1962–1963 school year. Across the South as a whole, a mere 1 percent of black school children were attending school with whites in 1964—a full decade after \textit{Brown}\index{Brown v. Board of Education} was decided.
\end{quotation}

In the years after \textit{Brown\index{Brown v. Board of Education}}, some localities went further than merely forming segregation academies. Prince Edward County in Virginia closed all of its schools for five years rather than integrate. Other jurisdictions closed pools, parks, zoos, and recreational facilities instead of integrating. This deliberate evasion of racial equality continued until a 1968 Supreme Court ruling put a stop to the practice of closing public facilities to avoid integrating them \parencite{Brennan1968}.

The irony is that while charter schools started life as 100\% white, they now serve intensely segregated students of color. \textcite[47]{Frankenberg.etal2019} noted that,

\begin{quotation}
Nearly three out of four students in the typical black student's charter school are also black. This indicates extremely high levels of isolation, particularly given the fact that black students comprise less than one-third of charter students. Latino isolation is also high, but not as severe as for blacks or whites across all charter schools.
\end{quotation}

Unfortunately, these segregation academies still exist, but instead of excluding children of color the way segregation academies did, they disproportionately target and enroll children of color. While these schools are no longer referred to as segregation academies, they make up a sizable subset of charter schools and often include the word ``Academy'' in their name. In Santa Clara County, for example, 11 out of 21 charter schools authorized by the county currently include ``Academy'' in their name \parencite{SCCOE2023}.

Nikole Hannah-Jones, in her keynote speech at the Network for Public Education's Fourth Annual Conference, said that it has never been the case that a majority of African-Americans have attended majority white schools \parencite{Hannah-Jones2017}. She then added ruefully, that this was quite a feat considering that African-Americans make up roughly one seventh of the population of the United States. \citeauthor{Orfield.Frankenberg2014} note that the percent of African-Americans in majority white schools rose from 0\% in 1954 to a peak of 43.5\% in 1988 before steadily declining to 23.2\% in 2011. \parencite[Table 3: Percent of Black Students in Majority White Schools, 1954–2011,][10]{Orfield.Frankenberg2014}. Hannah-Jones also commented that American public education does not even live up to the Separate but Equal doctrine espoused in \textit{Plessy v Ferguson} and overturned by \textit{Brown\index{Brown v. Board of Education} v Board of Education}. More recently, \citeauthor{Heilig.etal2019a} made the same point using 2015--16 Common Core of Data. They say, ``Nationally, we find that higher percentages of charter students of every race attend intensely segregated schools.'' \parencite[205]{Heilig.etal2019a}. This segregation has an effect on the achievement of the students thus segregated: it makes the ``achievement gap'' worse. %chktex 38

\begin{quote}
Racial segregation is strongly associated with racial achievement gaps, and the racial difference in the proportion of students’ schoolmates who are poor is the key dimension of segregation driving this association.  \parencite[47]{Reardon2016}
\end{quote}

\section{Charter Schools, Free Markets  and Privatization}\label{sec:freemarkets}\indent

Just a year after \textit{Brown\index{Brown v. Board of Education}},  \textcite{Friedman1955} published his article \citetitle{Friedman1955} in \citetitle{Solo1955} \parencite{Friedman1955} that reframed charter schools as an economic problem in education instead as a way of evading court-ordered integration. That paper ensured that charter schools would no longer be morally tainted by their association with virulent racism, but rather would take on the honorable task of breaking up what was called a monopoly. Charters, operating in a free market,\footnote{No one really wants a free market because a  market completely free of regulation would have unenforceable contracts, rampant monopolies, and constant and ruinous market failures. What people really want when they use the phrase ``free market'' is a heavily regulated market which allows them to profit, unfettered, while restraining or excluding others.} would allow parents to choose the best alternative from an array of competing choices. Tellingly left unspecified was exactly how the free market would ensure that the array of competing choices actually offered valuable educational alternatives rather than mere alternatives.

In 1981, Ronald Reagan ran and became President of the United States based on a platform of less government is better government. This platform included eliminating the U.S. Department of Education \parencite{gop1980}. True, eliminating the Department of Education is not the same as shutting down an entire school district the way white parents did in 1964, but the thought is there. \textcite{Haney-López2014} expertly dissects how it is possible to voice racist thoughts without actually using racial words, a practice perfected by President Ronald Reagan \parencite{Haney-López2014}.

Now, only liberty and freedom matter, in education, as in other fields. It is school choice or bust; school choice is proffered not only as \textit{the} panacea for all that ails America's schools, but it is even touted as the morally right thing to do. Without a  trace of irony, the former President Donald Trump framed school choice as the ``civil rights issue of our time'' in a garbled statement at the signing of an executive order on Safe Policing for Safe Communities:
\begin{quote}
School choice is the civil rights statement of the year, the decade and probably beyond. Because all children have to have access to quality education. A child’s zip code in America should never determine their future \parencite[as quoted in][]{Lennox2020}.
\end{quote} 

Education reformers have latched on to the notion that schools need to be privatized and freed from bureaucratic control for reasons of efficiency, increased flexibility, and accountability \parencite{Garcia2018}%
% [63]%
. This claim is made despite educational management organizations (EMOs) themselves being high overhead, opaque bureaucracies with scant accountability.

\citeauthor{Baker.Miron2015} identified four major policy concerns with the privatization of public education:
\begin{quotation}\noindent
  \begin{enumerate}
    \item A substantial share of public expenditure intended for the delivery of direct educational services to children is being extracted inadvertently or intentionally for personal or business financial gain, creating substantial inefficiencies;
    \item Public assets are being unnecessarily transferred to private hands, at public expense, risking the future provision of “public” education;
    \item Charter school operators are growing highly endogenous, self-serving private entities built on funds derived from lucrative management fees and rent extraction which further compromise the future provision of “public” education; and
    \item Current disclosure requirements make it unlikely that any related legal violations, ethical concerns, or merely bad policies and practices are not realized until clever investigative reporting, whistleblowers or litigation brings them to light.
  \end{enumerate} \parencite[3]{Baker.Miron2015}.
\end{quotation}

In California at least, these policy concerns have not been addressed in the six years since \citeauthor{Baker.Miron2015} wrote about them.\footnote{Changes in policy to address some of these concerns have been strenuously opposed by charter school advocates. For example, the California Charter Schools Association opposed an accountability bill, \textit{AB1316 School accountability: financial and performance audits: charter schools: contracts. (2021–2022)}, which merely sought to make charter school finances more transparent.}

Charter schools are now just one of the many forms of \textit{privatization}, when public functions are performed by private parties for profit. Privatization is a manifestation of the corporate takeover of the world, first documented more than fifty years ago by \citeauthor{Domhoff2014} and elaborated on in seven subsequent editions. \citeauthor{Domhoff2014} argues that corporations and the corporate elite really run the United States, and by extension, the world. \textcite{Kahn.Minnich2005} make much the same point in their book \citetitle{Kahn.Minnich2005} \parencite{Kahn.Minnich2005}. They list ``[s]chools, prisons, welfare, Social Security, water and sewer systems, buses, trains, subways, highways, waterways, sanitation systems'' (p. 30) as examples of formerly government run functions that are in whole or part privatized. They could have also listed postal mail, space travel, and now every facet of education, as being wholly or partly privatized. \textcite{Cohen.Mikaelian2021} lay out in detail how privatization has infiltrated American life and the consequences of this takeover of public goods by private firms run for profit  \parencite{Cohen.Mikaelian2021}. \citeauthor{Black2020} in \citetitle{Black2020} \parencite{Black2020} focuses on the less tangible but arguably more important consequences of privatization of public schools, the loss of democratic control.

Privatizers make money by turning goods or services that used to be publicly available into private goods and services that must be paid for before they can be used. The canonical example of privatization is the enclosure of the commons in Britain in the 16\textsuperscript{th} and 17\textsuperscript{th} centuries whereby land that previously had been owned collectively by a village was now owned by an individual who charged villagers for the privilege of using that land \parencite{SimonFairlie2009}. But modern privatizers have many more ways of turning a profit. They can obtain tax benefits, invest in other firms with public monies, invest in financial instruments with public monies, obtain a monopoly, engage in fraud, corruption, or outright theft, engage in self-dealing, obtain grants or loans on favorable terms, sell what does not belong to them, avoid paying for externalities, pay below market rates for goods or services, skew public-private partnerships to create unearned profits, engage in pay-for-success contracts, or offer social impact bonds.

Charter school operators have even more options. They can inflate enrollment, charge excessive management fees, mis-characterize expenses, omit or inaccurately report financial data, fail to open a school or close one soon after receiving a grant, or sell their facilities to investors and lease them back, all at potentially inflated prices. Many charter schools have a long history of duplicitous or fraudulent actions \parencite{ITPI2018, Burris.Bryant2020, Baker.Miron2015}.

School choice has been actively marketed and promoted by billionaires who do not send their children to public schools.\footnote{\textcite{Ravitch2016} lumps these billionaires together, calling them the ``Billionaires  Boys Club\index{Billionaires Boys Club}'', an epithet first used in \citetitle{Ravitch2016}.} The Walton family, Eli Broad\index{Broad, Eli}, Bill Gates\index{Gates, Bill}, the Koch brothers\index{Koch brothers}, the Zuckerbergs\index{Zuckerbergs}, and Laurene Jobs\index{Jobs, Laurene}, are all on the list of the 500 richest people in the world. Their collective wealth\index{billionaires, wealth} exceeds half a trillion dollars, and they are busily engaged using that wealth to fix the very problems that their accumulation of wealth caused. \textcite{Giridharadas2018} whose book, \citetitle{Giridharadas2018}: \textit{The Elite Charade of Changing the World}, says that it is a ``Trying-to-Solve-the-Problem-with-the-Tools-That-Caused-It'' issue \parencite[142]{Giridharadas2018}.

The effects of billionaire spending on education cannot be over emphasized. Bill Gates\index{Gates, Bill} made \$2B in grants aimed at creating smaller schools \parencite[11]{Gates2009}, despite a lack of evidence that they were educationally valuable. These grants were eventually discontinued when the initiative did not produce the intended results. Gates\index{Gates, Bill} was also instrumental in funding and promoting the Common Core State Standards\index{Common Core State Standards} and associated assessments whose premise was that if we only had high enough academic standards, student outcomes would improve, again without evidence that the reforms were educationally valuable and without evidence of a mechanism of improvement.

\section{Types of Charter Schools}\label{sec:types-charters}\indent

Charter schools can be broadly classified along three axes.\index{charter schools!classification of} The authorizer/oversight axis has to do with what entity approved their charter and who will exercise oversight. The profit/non-profit axis classifies schools by their intent to generate a profit, or not. Lastly, the in-person/blended/virtual axis characterizes pedagogical approach: Are their classes in-person, virtual, or a blend of the two?

\subsection{Charter School Authorizers and Oversight}\indent

Charter schools in California are potentially subject to a three step process to gain authorization to operate.\index{charter schools!authorization process} The first step is to submit a petition to the school district in which the charter wishes to operate. This petition must contain a number of required elements, all of which are specified in Education Code §47605(c)(5)(A–O), the commonly called ``15 Required Elements (A-O elements)'' \parencite[89]{Aguinaldo.etal2021}.\index{charter schools!petition contents} Besides some technical details, the petition must contain a description of the charter's annual goals which must align with state priorities, for all pupils and for various subgroups; how these outcomes are to be measured; how the charter is to achieve a racial and ethnic balance similar to its district, its governance structure, and its finances. All of these elements are captured in \textit{\citetitle{FCMAT2022}} by the Fiscal Crisis and Management Team (FCMAT), a document intended to provide a legally sound checklist for authorizers \parencite{FCMAT2022}.\index{charter schools!authorizer checklist}  % chktex 36

If a petition contains all the required elements, then the public school district may approve the petition, possibly with additional stipulations. If the public school district denies the charter school's petition, it must state why. The charter school may appeal that denial to that County's Board of Education (CBOE), and if the CBOE denies the charter school's appeal, under certain circumstances, the charter school may appeal to the State Board of Education (SBE). A denial by the SBE terminates the process, and the charter school is not permitted to open.

\index{charter schools!authorizers|(}
Public school districts (LEAs, local education agencies, in the parlance of the California Department of Education (CDE)) may authorize one several kinds of charter schools. \prettyref{tab:school_attributes} is a summary of the attributes of the types of schools in California. A public school district may sponsor a charter school directly, in which case the district exercises oversight. These dependent charter schools are authorized by the local public school board and are subject to the board's jurisdiction. It also is possible for all the schools in a district to convert to charter schools, and then the public school board becomes the charter school board. Lastly, charter schools may be authorized by a public school district or a county office of education with a governing board that is distinct and independent from the authorizer's governing board.
\index{charter schools!authorizers|)}

\begin{table}[ht]
  \caption[Attributes of Private, Charter, and Public Schools in California]{\textit{Attributes of Private, Charter, and Public Schools in California}}\label{tab:school_attributes}\index{schools!attributes}%
  \begin{tabular}{lllll}\toprule
    \textbf{Attribute}  & \textbf{Private}    & \textbf{Charter} & \textbf{Public}  \\\midrule
    Funding               & parent tuition      & tax dollars      & tax dollars      \\
    Governance            & self-appointed      & self-appointed   & elected board    \\
    Duration              & unlimited           & time-limited     & unlimited        \\
    Ed. Code              & no                  & no               & yes              \\
    Taxation Powers       & none                & none             & limited          \\
    Facilities Bonds      & no                  & no               & yes              \\
    Facilities Grants     & no                  & yes              & no               \\
    Enrollment            & limited             & limited          & not capped       \\
    Unionized             & rarely              & rarely           & often            \\
    Curriculum            & completely flexible & very flexible    & flexible         \\
    Standardized Testing  & no                  & yes              & yes              \\
    Accountable           & no                  & authorizer       & elected board    \\
    Teacher Certification & no requirement      & yes              & yes              \\
    Teacher Pension       & perhaps             & perhaps          & yes              \\\bottomrule
  \end{tabular}
\end{table}

\subsection{Profit-Making Status}\indent

\index{charter schools!profit-makeing|(}
Until the 2019–20 school year, charter schools in California could be run directly or indirectly by a profit-making organization. California now prohibits profit-making organizations, either a single school or a charter management organization, from submitting an initial charter school petition or a renewal.

Even though profit-making charters are banned, there are many ways of getting around this restriction. Charter operators can contract with outside firms to provide all or just some services, and those firms may be profit-making firms. Charter operators are able to lease, buy, or sell their facilities, and those transactions might generate a profit. Charter operators can sell their facilities and lease them back from the buyer. This kind of financial transaction converts an illiquid asset (buildings) into a liquid asset (cash) and also generates a revenue stream from the rental income, all of which is ultimately paid for by taxpayers. Charter operators may also charge schools a management fee or an expansion fee. Charter operators are not restricted in the salaries they pay administrators.
\index{charter schools!profit-makeing|)}

\index{charter schools!conflict-of-interest laws|(}
However, charter school board members in California have recently become subject to the conflict-of-interest laws specified in Government Code §§1090–1099 and §§87100–87314 \parencite{Becerra.Medeiros2018}. Generally, government officials are prohibited from benefiting financially from their positions as public servants, but it remains to be seen if these conflict-of-interest laws will prevent profiteering by school board members, administrators, or relatives of either.\footnote{The law is necessarily complex. Two useful guides (total: 300 pages) are \textcite{Chaney.etal2010} and \textcite{Ennis.etal2016}. A more general guide to local government ethics is \citetitle{InstituteForLocalGovernment2016} from California's \citeauthor{InstituteForLocalGovernment2016} \parencite{InstituteForLocalGovernment2016}.}
\index{charter schools!conflict-of-interest laws|)}

\subsection{Types of Instruction}%
\label{sec:types-instruction}\indent%

\index{charter schools!pedagogy!similarity to public schools|(}
Charter schools, unlike almost all public schools, vary in their instructional format. In-person instruction is similar to that in traditional public schools, with one exception: the so-called ``no excuses'' charter schools \parencite{Horn2016, Torres.Golann2018, Golann2021}. These schools emphasize a highly scripted, rigid code of conduct that relies on fear, intimidation, and Skinnerian behavior modification as foundational elements of their pedagogy.  Unlike schools which offer in-person instruction, virtual charter schools have no face-to-face instruction; everything is mediated by some sort of technology, typically, computers running specialized software, paid for by taxpayers. In between in-person instruction and virtual instruction is blended learning. It is simply a mixture of in-person and virtual instruction \parencite{Horn.Staker2015}.\index{charter schools!pedagogy!similarity to public schools|)}

\index{charter schools!virtual|(}
Since 2013, virtual charter schools have been studied extensively by Alex Molnar, Gary Miron and others and at the National Education Policy Center, University of Colorado, Boulder \parencite{Molnar.etal2013, Molnar.etal2014, Molnar.etal2015, Miron.Gulosino2016, Molnar.etal2017, Miron.etal2018, Molnar.etal2019, Molnar.etal2021}. Their annual reports are depressingly consistent: virtual schools not run by a public school district significantly underperform public schools. Their conclusions are echoed by \textcite{Woodworth.etal2015, Garcia2018}. Yet, despite being clearly academically inferior to public schools, the number of students attending virtual schools has risen year after year. Their pre-pandemic growth seems to be slowing, but their performance, compared to public schools, has not measurably improved.\footnote{Although \citetitle{PublicAgenda2018} is otherwise an excellent summary of the research on charter schools, they incorrectly state (p.117) that there is little research of online or virtual charter schools. The authors must not be aware of the NEPC series on virtual charter schools. However, according to \textcite[117]{Molnar.etal2019},there is only one study on blended charter schools.}
\parencite[11]{Molnar.etal2019}.\index{charter schools!virtual|)}

Pre-pandemic, charter schools in California were legally deemed classroom-based (e.g.~not virtual) if students spent no more than 20\% of their time in front of a computer.\footnote{The California Education Code §47612.5(e)(1) does not mention computers, but bases its definition of classroom-based on students being physically at the schoolsite with a certificated teacher in charge. Under that definition, a roomful of students behind computers with a teacher in attendance would qualify as classroom-based and not virtual. California's Education Code does not recognize the blended category.}  Blended charter schools, on the other hand, offer some sort of face-to-face interaction with a teacher along with online activity without face-to-face interaction. But they too offer only marginally better educational outcomes than fully virtual charter schools \parencite%.%chktex 36
%[52]%
{Molnar.etal2019}. Rocketship schools use a blended instructional model.

\section{Charter Schools in the United States}\label{sec:us-charter-schools}\indent

Charter schools are one of several different kinds of school choice that are or have been available in the United States. Vouchers, private schools, home schooling, educational savings accounts, freedom-of-choice plans, magnet schools, and open enrollment are all forms of school choice. Home schooling accounts for less than 5\% of all the students in  United States. Private schools enroll about 12\% of the total. Magnet school account for a few percent. Roughly, the various form of school choice, including charter schools, account for just under a quarter of all American students.

The characteristic that home schooling and private schools share is that they are agnostic about public schools. Not so for charter schools, voucher, and freedom-of-choice plans. Charter schools, voucher programs, parent trigger programs, and freedom-of-choice plans explicitly want to supplant or replace public schools \parencite{Garcia2018}.
%pp. 5,15,35%

The first charter schools, other than segregation academies, were founded in Milwaukee, Wisconsin in 1991, followed by California starting in 1993. Conceptually, charter schools were based on an amalgam of ideas from Milton Friedman,\index{Friedman, Milton} Albert Shanker,\index{Shanker, Albert} and Ray Budde.\index{Budde, Ray} Milton Friedman came at it from an ideological point of view couched in economic terms. Albert Shanker, in 1988, in a speech at the National Press Club, proposed that \textit{teachers} in conjunction with \textit{parents} be allowed to form a school \textit{within} a school district. Shanker made no mention of competition, or free markets, or even of charter schools. Shanker's speech emphasized curriculum and learning, not governance or finance.\index{charter school!origial concept} Ray Budde first thought of charter schools in the early 1970s, but his proposal generated no interest and it was not until 1988 that he published his ideas \parencite{Budde1988}.

\subsection{Charter Schools in California}\label{sec:charters-in-ca}\indent

Charter schools, in California as elsewhere in the United States, enter into a contract (the charter) with a chartering authority that specifes what they are to do and how, and in return, are exempt from the entirety of California's Education Code (with the exception of five technical provisions). The California Legislature, when it enacted the \textit{The Charter School Act of 1992}\footnote{Current California law can be accessed at \url{https://leginfo.legislature.ca.gov/faces/home.xhtml}. California Regulations are at \url{https://ccr.oal.ca.gov}. California's Education Code (Ed.Code) is at \url{https://leginfo.legislature.ca.gov/faces/codesTOCSelected.xhtml?tocCode=EDC&tocTitle=+Education+Code+-+EDC}} (Ed. Code §47600), spelled out its intent in passing that legislation.\index{Charter School Act of 1992, intent of} The Act has been amended a number of times in its nearly 30 years of existence, but its intent has remained the same. It specifies that charter schools should

\begin{enumerate}[label=\alph*)] %chktex 9 %chktex 10
  \item Increase learning opportunities for all pupils, with special emphasis on expanded learning experiences for
  pupils who are identified as academically low achieving.

  \item Create new professional opportunities for teachers, including the opportunity to be responsible for the learning program at the school site.
  \item Provide parents and pupils with expanded choices in the types of educational opportunities that are available within the public school system.
  \item Hold the schools established under this part accountable for meeting measurable pupil outcomes, and provide the schools with a method to change from rule-based to performance-based accountability systems.
  \item Provide vigorous competition within the public school system to stimulate continual improvements in all public schools.\footnote{This goal was added in 1998.}
\end{enumerate}

It is important to keep these goals in mind because charter schools have contractually agreed to meet these goals in return for funding, independently of whatever other goals they explicitly specified in their initial petition. Note, in particular, that the Legislature said nothing about profitability, and in fact, California enacted in 2018 a prohibition against for-profit charter schools (Ed. Code §47604 et seq.).

\section{Surveys of Charter School Research}\label{sec:charter-surveys}\indent

\index{charter schools!surveys of research on|(}
It has been about 30 years since the first charter school law was passed. In the last decade, researchers have published several surveys of the research on charter schools. The first two decades (~1990–2010) were somewhat experimental and different enough that the research that came out of that period is less relevant than more recent research. The first survey of the last decade, is \citetitle{Smith.etal2011}.  In it, \textcite{Smith.etal2011} reviewed systematically charter school research as it existed in 2011. The authors were interested, not so much in the conclusions of the studies they looked at, but how the research was performed, how it was structured, what facets of charter schools were examined, and what was the subject of the research in order to ``separate empirical evidence from politicized conjecture'' (p. 460). They reviewed a total of 323 peer-reviewed articles and research center reports and found that student and school outcomes were the most commonly studied topics. They noted many studies were unable to generalize their findings because of variations in policy between states and localities. The authors also noted that there was a lack of longitudinal studies which is not surprising due to policy variations. Furthermore, they found that ``acceptance into a peer reviewed journal does not always ensure that qualitative research adheres to the standards of providing substantiation that findings are credible and trustworthy or that quantitative research provides evidence of the studies' validity, reliability and generalizability.'' (p.466) Finally, the authors noted that many studies could not draw causal connections.

Four years later, \textcite{Berends2015} chose as his focus the various theories that researchers used when looking at the social organization of charter schools. In \citetitle{Berends2015}, \textcite{Berends2015}, found, like \citeauthor{Smith.etal2011}, that most studies concentrated on student achievement and neglected educational attainment such as high school graduation, college admission, and the granting of a degree. He notes that ``the effects of charter schools on student achievement are mixed (some positive, some negative and some neutral)'' (p. 170) \citeauthor{Berends2015} thinks the context in which charter schools operate is important in order to understand the magnitude of any effects and to understand what we can expect from school reform. He identifies longer school days, a focus on achievement, behavioral policies, teacher coaching and feedback, and data-based decision-making as characteristics most often associated with effective charter schools. Lastly he looks at innovation and distinguishes between curriculum and class-room based changes, and organizational changes, and he found hat charter schools mostly innovate on the structural side rather than the academic side. 

Next, \textcite{Epple.etal2016}, in \citetitle{Epple.etal2016}, did much the same as \citeauthor{Berends2015}, but concentrated on the technical aspects of study design \parencite{Epple.etal2016}. The authors observed that which the research question being answered by a particular study was often much narrower or significantly different than the research question authors set out to answer or thought they were answering. The heart of their review is an analysis of ``the methodological challenges in evaluating charter effectiveness'' (p.141), and the strength and weaknesses of the various approaches that have been used. They find that researchers used one of five statistical methods: lottery-based design, fixed-effect approaches,  matching procedures, ordinary least squares (OLS) regression, and instrumental variable approaches (p. 165), and they evaluate each approach. \citeauthor{Epple.etal2016} also discuss the much scrutinized virtual control record method of matching charter school students to public school students that came out of Stanford's Center for Research on Education Outcomes (CREDO) which was criticized on purely statistical grounds in \textcite{Gabor2015}. 

In 2015 and then updated in 2018, \citeauthor{PublicAgenda2018} released a guide to charter school research for non-academics, a review of current charter school research that was written in a way that is accessible to the public. The chapter on finance focused on four questions: how charter schools are funded, how charter schools and traditional schools compare in per pupil funding, what financial effects do charter schools impose on traditional public schools, and what are, if any, differential spending patterns between traditional public and charter schools \parencite[78–89]{PublicAgenda2018}.

The finance chapter revealed that the 48 states with school choice programs had 48 different methods of funding public schools and charter schools. This variation in funding models made comparisons difficult. In addition, each state has likely gone through several iterations of models of charter school funding, and this lack of commonality prevents researchers from conducting valid longitudinal studies. The authors answered their first question on funding by referring to a compilation of state funding amounts.

Their answer to the second question was yes, different levels of funding do exist, and in a few cases, by as much as 40\% to nearly 60\% less. Their take on whether it matters was hedged because studies differ in their conclusions for a variety of reasons. Likely not published in time to be reviewed by \citeauthor{PublicAgenda2018}, was \textcite{Baker2018} which emphatically says that money does matter. They answer their third question with an unambiguous yes, charter schools do affect the finances of public schools. More recent research, \textcite{Lafer2018}, \textcite{Baker2019},  and \textcite{Miron.etal2021} validates their conclusion. Finally, they conclude that charter schools do spend their revenues differently, in part because charters spend more on administration than public schools do and sometimes more on facilities. 

The last of the four academic surveys, \textcite{Zimmer.etal2019}, considers who was served, racial segregation effects, both academic and non-academic outcomes, management structure, and financial effects of charter schools. Since Zimmer is a co-author of both this survey and of the previously cited \textcite{Epple.etal2016}, the kinds of study designs analyzed are similar. \citeauthor{Zimmer.etal2019} intend to synthesize ``the best research to inform the debate [about the value of charter schools]'' (p. 2). They go beyond the 2016 study and survey studies on racial segregation, selective enrollment, and student pushout. \citeauthor{Zimmer.etal2019} conclude that charter schools lead to greater segregation for African Americans, but not necessarily for whites or Latino students. They find that charter schools do engage in sometimes subtle forms of selective enrollment and student pushout. Independently, and two years later, \citeauthor{Mommandi.Welner2021} document thirteen major ways that charter schools effectively choose who they enroll \parencite{Mommandi.Welner2021}. After summarizing three different kinds of research (fixed effects, lottery-based, and match and other regression), they turn their attention to research on non-cognitive outcomes. Their penultimate chapter looks at research on indirect effects.  

Although \textcite{Garcia2018} is not explicitly a survey of the existing literature, it contains in Chapter 3 
%, (pp. 91–146),
much material on the research evidence which guides (or should guide) school choice policies. His goal is to present general trends that ``reflect the weight of the evidence'' (p. 93). The weight of the evidence, Garcia finds the research points to the conclusions that ``school choice policies are more likely to separate, rather than integrate, students from different racial/ethnic and socioeconomic backgrounds'' (pp. 159–60), ``how countries and states structure school choice policies can have a profound impact on how school choice functions at a practical level'' (p. 160), ``low-income students face obstacles to participating in school choice plans'' (p. 161), lastly, ``one should expect student achievement gains under school choice plans to be modest at best and inconsistent across subjects and years'' (p. 161), and ``a major reason for the inability of school
choice to have an impact on the academic core of schools—teaching and learning—is that school choice came of age
at the same time as high-stakes accountability policies that encourage standardization'' (p. 162)

Garcia makes a point that had not been made before: Since both public schools and charter schools are measured the same way (standardized tests), ``the incentives to implement innovative pedagogical strategies are curtailed because the methods by which students are able to demonstrate their learning are uniform across all schools and restricted to the format of the tests.'' (p. 163) He predicts that school choice in its many forms will continue to expand. %chktex 38
\index{charter schools!surveys of research on|)}

\subsection{Research on Charter School Finances}\label{sec:rese-chart-scho}\indent

Charter schools have been much studied, and the last decade has produced a number of reports examining charter school finances based on carefully collected evidence. For example, in 2014, \textcite{Lafer2014}, now at In the  Public Interest, published an analysis of a proposed law in Milwaukee, WI \parencite{Lafer2014} that was specifically tailored to benefit a to-be-opened Rocketship school. Lafer went on to author two other studies on charter schools, public policy, and finance: \citetitle{Lafer2017} \parencite{Lafer2017} and \citetitle{Lafer2018} \parencite{Lafer2018}. Carol Burris, Executive Director of the Network for Public Education, and several co-authors have produced three reports on money and charter schools: \textcite{Burris.Pfleger2020}, \textcite{Burris.Bryant2020}, and \textcite{Burris.Cimarusti2021}. The National Education Policy Center, a research center based at the University of Colorado, Boulder, with over 150 scholars and academics from institutions across the U.S. whose goal is ``to produce and disseminate high-quality, peer-reviewed research to inform education policy discussions'' \parencite{NEPC2021}, has produced hundreds of reviews of research, policy and legislative briefs, some of which are annual surveys of charter schools. The series on profiles of EMOs have been produced annually for fifteen years; the series on virtual charter schools, for ten years.

Baker's contributions to the NEPC are especially noteworthy. He is an author or co-author of 28 reviews of reports, studies, or articles on school finance, in addition to six policy, legislative, or research briefs. Baker co-wrote with Gary Miron \citetitle{Baker.Miron2015} \parencite{Baker.Miron2015} which introduces many of the tools and techniques for evaluating how charter schools operate for profit.

\textcite{Lafer2017}'s report, \citetitle{Lafer2017} is particularly revealing. He writes,
``Any time there is a low bar of entry for firms seeking to access government funds, one can expect to find corruption, and the charter industry is no exception.'' (p.18) But even absent corruption, there is ample opportunity to make lots of money. Lafer documents \$2.5B of Californian taxpayer money spent over fifteen years on charter school facilities, in many cases where there is no documented educational need and where the charter school is of lower quality that nearby public schools. Lafer says, ``It's as if legislators turned on a faucet of money and then just walked away.'' (p.12) Since Lafer's report came out four years ago, only half-hearted changes have been made to turn the faucet off.\footnote{In California, new for-profit charter schools have been outlawed, and it has been established that charter schools must abide by conflict of interest laws. Existing, for-profit charter schools may renew their charters, and using a for-profit charter management organization to ``sweep'' all of a charter's revenue allows charters to bypass any conflict of interest law. Consideration of the financial impact of a charter school on its host district is now allowed, but only for initial petitions. Finally, the ability of the State Board of Education to ignore a district's or a county's denial of a petition has, in most cases, been eliminated.}

\section{Rocketship}\label{sec:rocketship}\indent

Rocketship is well-known in the charter school world. It even has been the subject of a ``biography'', \citetitle{Whitmire2014} \parencite{Whitmire2014}.\footnote{Just three other charter schools share this distinction: Geoffrey Canada's Harlem Children's Zone \parencite{Tough2009}, \citeauthor{Jacobs2007}'s \citetitle{Jacobs2007}, and the KIPP schools \parencite{Mathews2009, Horn2016}}  Rocketship's leaders and supporters routinely describe it as ``high performing'', ``deserving of huge credit'', ``dynamic'', and ``nationally lauded''. Rocketship schools, it is claimed, outperform some of the best public schools in the country. Rocketship ``believe[s] that every student deserves the right to dream, to discover, and to develop their own unique talent''.\footnote{Rocketship, like many charter school advocates and privatizers, excel at choosing memorable, compelling names and tag lines that are impossible to argue against but which nonetheless misrepresent — deliberately so — their goals.}

Rocketship is one of the largest non-profit charter school chains in the United States. They operate 21 schools in the United States; thirteen in California, three in each in Nashville, TN and Washington, D.C., and two in Milwaukee, WI.~In Santa Clara County, CA, they have eight TK-5 elementary schools authorized by the county that served 4,254 students in the 2019–20 school year plus 1,240 students in two district authorized schools, for a total of 5,494 students.

\subsection{Rocketship History}\label{sec:history}\indent

On February 16, 2006 John Danner, a tech entrepreneur turned educator, filed with the California Secretary of State incorporation papers for Rocketship Education. Danner, Don Shalvey, Jennifer Andaluz, and Eric Resnick are listed as the initial members of Rocketship Education's board of directors. Danner had significant teaching experience (Nashville, TN public schools) prior to Rocketship, as did Shalvey (Aspire Public Schools) and Andaluz (founder and Executive Director of Downtown College Prep, a charter school in San José). Resnick, the fourth member in the founding group, was a hedge fund manager who had a ``a deep understanding of financial management and real estate transactions'' \parencite[13]{Danner2006}. The inclusion of Resnick, an expert in real estate transactions, at the very beginning of Rocketship, is interesting because one of the preferred ways for charter school investors and founders to generate profits is via real estate deals. John Danner eventually left Rocketship in 2013 to found Zeal, an online math tutoring tool, and was replaced by Preston Smith who became CEO\@. Smith became the first principal of the Rocketship's first school, Mateo Sheedy, and was subsequently listed as a Rocketship co-founder in the charter petition for Rocketship's second school. % chktex 13

Matt Hammer, Executive Director of PACT (People Acting in Community Together), brought Danner and Smith together, and has promoted charter schools through his advocacy non-profit, Innovate Public Schools.\footnote{\url{https://innovateschools.org/}} Reed Hastings was an early supporter of Rocketship and he proselytized Rocketship to the larger charter school community. When he promised Rocketship \$250K for each of the first eight Rocketship schools opened, his donation caught the attention of philanthropic venture funds \parencite[50]{Whitmire2014}. 

Danner chose to open his first school in the San José Unified School District. Danner prepared a 300 page petition that he submitted to the San José Unified School District on 04 May 2006, which held a public hearing on the matter on 20 June 2006. On 13 July 2006, the SJUSD Board of Education denied his petition, again at a public meeting. He then appealed this denial to the Santa Clara County Board of Education and presented a modified petition that – as far as the SCCBOE was concerned – overcame the objections raised by the SJUSD\@. They conditionally approved Danner's first charter school at their 18 October 2006 meeting. The school opened in August 2007 for the 2007-08 school year.

As of 2023, Rocketship had expanded to 23 schools in California, Tennessee, Wisconsin, Washington, D.C. and Texas, ten of which are in Santa Clara County. \prettyref{tab:RocketshipSchools} on page \pageref{tab:RocketshipSchools} lists those ten, when they opened, and when they submitted initial petitions and renewal petitions.

Opening schools did not go smoothly for Danner and Rocketship. Not only was there community opposition, but various community organizations also opposed opening one or more Rocketship charter schools. The most consequential opposition was a 2014 lawsuit brought by the Alum Rock, Evergreen, Franklin-McKinley and Mount Pleasant school districts which contended that the SCCOE had exceeded its authority in approving in advance 20 county-wide Rocketship charters, bypassing local school districts as authorizers. At the time of the lawsuit, three Rocketship charters had opened under this county-wide authorization, and in a settlement, Rocketship agreed not to seek to open 13 of the 20 charters. In the end, only five county-wide charters opened.\footnote{Sharon Noguchi reported that nearly \$500K was spent on this lawsuit and not on educating children \parencite{Noguchi2015}.}

Over a period of nine years, Rocketship opened ten schools in Santa Clara County. Eight schools were either countywide charters or charter schools whose petitions were denied by the local public school district, but subsequently approved by the Santa Clara County Board of Education. \prettyref{tab:RocketshipSchools} lists the eleven Rocketship schools that were approved and the ten that opened. Note that only two schools were approved by the school district in which there were expected to locate. This lopsided result suggests that current charter school laws are tilted in favor of charter schools.

\begin{table}[ht]
  \caption[Rocketship Schools in Santa Clara County, California]{\textit{Rocketship Schools in Santa Clara County, California}}%
  \label{tab:RocketshipSchools}\SingleSpacing\small%
  \begin{tabular}{lllll}\toprule
    \textbf{School}  & \textbf{Type}    & \textbf{Opened}  & \textbf{Renewed} & \textbf{Notes}                \\
    \midrule
    Mateo Sheedy    & District appeal & 2007            & 2009, 2015, 2019 & \multirow[t]{2}{1.5in}{Denied by SJUSD, approved by SCCBOE}\\\\
    Sí Se Puede     & District appeal & 2009            & 2011, 2017       & \multirow[t]{2}{1.5in}{Denied by ARUSD, approved by SCCBOE}\\\\
    Los Sueños      & Countywide      & 2010            & 2015             & SCCBOE countywide charter\\
    Discovery Prep  & Countywide      & 2011            & 2016             & SCCBOE countywide charter\\
    Mosaic          & District        & 2011            & 2016             & Approved by ARUSD\\
    Brilliant Minds & Countywide      & 2012            & 2017             & SCCBOE countywide charter\\
    Alma Academy    & Countywide      & 2012            & 2017             & SCCBOE countywide charter\\
    Spark Academy   & District        & 2013            & 2018             & Approved by FMSD\\
    Alum Rock       & District appeal & —               &                  & \multirow[t]{3}{1,5in}{Denied by ARUSD, approved by SCCBOE, but withdrawn 2015}\\\\\\
    Fuerza          & Countywide      & 2014            & 2018             & SCCBOE countywide charter\\
    Rising Stars    & District appeal & 2016            & 2021             & \multirow[t]{3}{1.5in}{Denied by FMSD, approved by SCCBOE}\\\\\bottomrule
  \end{tabular}
\end{table}

\section{Rocketship Finances}\label{sec:rocketship-finances}\indent

Charter schools have a number of unique financial needs. They need startup funds, operating funds, and often funds to expand, funds that public schools do without. Rocketship is no exception. The \textit{operation} of charter schools are funded by federal, state, and local governments, but funding \textit{expansion} may or may not be funded with tax dollars, depending on the laws of a particular state. The difference between what is funded at taxpayer expense and what's not must somehow be funded with outside money. Startup money is needed for facilities, desks and chairs, teacher and administrator salaries, legal fees, curriculum materials, etc., all of this before even one student registers. Startup facilities cost vary widely. If the charter school chooses to use public school district facilities under Proposition 39\footnote{Proposition 39, passed by California voters in November 2000, contains a provision that requires public school districts to provide charter schools facilities ``sufficient to accommodate the charter school’s students'' \parencite[38—41]{sos.ca2000} \parencite{Prop39.2000}. Regulations governing Prop. 39 facilities are in California Code of Regulations, Title 5, §11969.}, their need for funds will be lower than if they choose to lease or build their own facilities. Startup facilities costs might involve the purchase of land and the construction of school buildings, or might just involve lease payments. But since state funding is tied to attendance, some startup funding is necessary. Thus the federal government provides grants, administered by the states, for this purpose.

Early on, Rocketship has indicated its intent to expand. In 2009, Rocketship announced plans to open six new schools \parencite{Cook2009}. It submitted a petition to Santa Clara County to open countywide charters and within three years had actually opened four schools. Like many other CMOs and EMOs, Rocketship needs to expand to increase revenue enough to be worth the while of investors. A single school's profit is not enough, but by using economies of scale, a ``portfolio'' of charter schools might suffice. A portfolio of charter schools\index{charter schools!portfolio of} is a collection of schools – almost always charter schools – managed as a whole.

The idea of a portfolio of schools comes from finance where a carefully chosen portfolio of investments can have lower risk collectively for a given level of return than a mere assembly of individual investments. \parencite[See ][for an overview of the mathematics of modern portfolio theory]{MarkowitzContributors2024}. \citeauthor{Hill.etal2009} claim to have invented the term \textit{portfolio school district} \parencite[1]{Hill.etal2009} and with it a strategy to implement such a district. Just a year later, \citeauthor{Henig.etal2010} defined portfolio strategy for schools as
\begin{quotation}\noindent
[A] loosely coupled conglomeration of ideas held together by the metaphor of a well-managed stock portfolio and its proponents’ \textit{unshakable belief} that the first step for successful reform must be to dismantle the bureaucratic and political institutions that have built up around the status quo [emphasis added] \parencite{Henig.etal2010}
\end{quotation}

\citeauthor{Hill.etal2009} acknowledge, in dry, understated language, that overcoming the objections and criticisms of educators and scholars to their unshakable belief will be difficult: ``It is hard to imagine that a portfolio strategy could be introduced into a major city without significant conflict.'' (p.2) Portfolio strategy is most often associated with Paul Hill and The Center for Reinventing Public Education, which is now located at the Mary Lou Fulton Teachers College at Arizona State University.

\subsection{Rocketship Expansion Funding}\label{sec:rocketship-expansion-funding}\indent

California, startup charter school funding has waxed and waned, in part because federal funding has varied. Currently, the U.S. Department of Education provides startup funds to states under the Charter Schools Program State Educational Agency (SEA) grant program\footnote{\url{https://www2.ed.gov/about/offices/list/oii/csp/funding.html}}. The federal charter school funding programs are listed in \textcite{NCSRC2020}. \citetitle{NAPCS2020} notes that 

\begin{quotation}\noindent
At the core of the Charter Schools Program are the Grants to State Entities (SE Grants). The State Entity program offers competitive grants to states, which then make subgrants within their states to \textit{open new charter schools and replicate or expand existing charter schools} [emphasis added] \parencite{NAPCS2020}.
\end{quotation}

Funds like the NewSchools Venture Fund\footnote{\url{https://www.newschools.org/}} and the Charter School Growth Fund I \& II\footnote{\url{https://chartergrowthfund.org/}} exist to fund the development and expansion of charter schools and charter management organizations. In 2007, when Rocketship Mateo Sheedy was started, Rocketship used lines of credit and loans to fund its beginning \parencite[260]{Danner2006}. Now, charter schools have many more options for funding startup or operations.

Charters have at least three other sources of facilities funding: bonds, tax credits and foundation or individual contributions. Betsy DeVos, who served as Secretary of Education for Donald Trump, has donated \$12.6M to Rocketship. Reed Hasting, a founder and now CEO Netflix has donated more than \$2M. In addition, charter schools can avail themselves of the New Market Tax Credit if they meet certain investment criteria, and if they do, they can get back 39\% of their investment  in tax credits in seven years. If their investment returns, say, 20\%, then combined, they are looking at nearly a 60\% return on their investment. A sixty percent return is fantastic. Charter schools and charter school operators can also issue revenue bonds. Revenue bonds are guaranteed by a revenue stream instead of by property tax revenues the way general obligation bonds are. Note that both are tax-exempt. As of 2015, charter schools issued over \$11B in revenue bonds according to \textcite{Clark-Herrera.etal2019}.

\subsection{Rocketship Expansion Difficulties}\label{sec:rocketship-expansion-difficulties}\indent

In 2014, the Santa Clara County Office of Education and Rocketship were sued by four Santa Clara County public school districts: Alum Rock\index{Alum Rock}, Mount Pleasant, Franklin-McKinley and Evergreen. At issue was the SCCBOE's bulk authorization of twenty countywide Rocketship charter schools. Sixteen months, 17,500 pages of evidence, and an estimated \$435,000 later, Rocketship, the public school districts, and Santa Clara County settled \parencite{Noguchi2015}. As part of the settlement, Rocketship agreed to withdraw 13 of the 20 countywide charters thus far authorized. Since one of the remaining countywide charter had already been withdrawn, that left six potential charters still authorized but as of yet, unopened. So far, it appears that Rocketship has instead attempted to expand in locations beyond Santa Clara County: San Pablo\footnote{unsuccessfully} and Concord in California, Nashville in Tennessee, Milwaukee in Wisconsin, Washington, D.C. and Fort Worth in Texas.

\subsection{Charter School Accountability}\label{sec:charter-accountability}\indent

In California, all K–12 schools, including privately managed charter schools like Rocketship, must submit annual budgets, Comprehensive Annual Financial Reports (CAFR), and since 2014, Local Control and Accountability Plans (LCAP). LCAPs are three year plans, possibly updated in years two and three, which detail how a school will use its funds to address state priorities, and to improve educational outcomes for foster youth, English learners, and low-income students, along with the metrics which will be used to show progress \parencite[66–84]{Aguinaldo.etal2021}. Given that the content of LCAPs are discretionary, and given that Rocketship specifically targets areas where the state is especially interested in improving educational outcomes, further analysis of LCAPs and budgeting may be warranted.
\section{Rocketship and Privatization}\label{sec:rocketship-privatization}\indent

Some contend that the central purpose of charter schools is to disguise a money-making operation \parencite{Saltman2018c}. \citeauthor{Whitmire2014}, who now sits on the board of Rocketship Education and who in 2014 published \citetitle{Whitmire2014}, makes note of the role that private venture funds played in Rocketship financing \parencite%
%[25,65]%
{Whitmire2014}, and it is instructive to remember that private, for-profit venture funds exist to make money. True, they often are ``double bottom line'' grantors \parencite{Clark.etal2004}. As \citeauthor{Ball2012} (cited in \textcite[75]{Tewksbury2016}) makes clear
\begin{quotation}\noindent
[P]articularly with the added case of Rocketship, a blended learning chain of charter schools, is that the NSVF [NewSchools Venture Fund] is using its clout to further blur the lines between for-profit and non-profit educational projects and organizations, thus smoothing the groves [grooves?] for marketizing educational policy and practices. Ball (2012) makes the connections and rationalities clear: ``Symbolically, philanthropy provides an `acceptable' alternative to the state in terms of its moral legitimacy.  It has also provided a kind of rehabilitation for the forms of capital that were subject of `ill repute' in the public imagination. Strategically, philanthropy has provided a ``Trojan horse'' for the modernizing move that opened the `policy door' to new actor and new ideas and sensibilities.'' \parencite[32]{Ball2012}
\end{quotation}

Privatizers use investment banks, hedge funds, and private equity firms as vehicles for investing \parencite{Stowell2018}. These investment vehicles are called \textit{alternative investments}, in contrast to \textit{traditional investments} like stocks and bonds. Investment banks provide the financial expertise that hedge funds and private equity firms need. 

\subsection{Privatization}\label{sec:privatization}\indent

Charter CMOs and EMOs appear to be following the lead of prison and health care privatizers. They lobby legislators intensively. They position themselves as being more efficient than the ``wasteful'' public sector, and they claim to to be able to do better that public schools, prisons or hospitals at a lower cost. Since charter schools have positioned themselves as being in competition with traditional public schools, they need to do at least as well as traditional public schools, or failing that, appear to do so. This calls for creative marketing, and so, to that end, pro-charter advocacy organizations, some university-affiliated institutions, and some think tanks have been harnessed to churn out pro-charter puff pieces that are regularly debunked.\footnote{The National Educational Policy Center (\url{https://nepc.colorado.edu}) in the School of Education at the University of Colorado (Boulder) currently has over 150 NEPC Fellows who aim ``to produce and disseminate high-quality, peer-reviewed research to inform education policy discussion'' on a wide variety of topics. They often review pro-charter school publications which have been presented as academic research even though those publications have not been peer-reviewed and often have serious methodological problems which weaken or negate their conclusions.} Evidently even creative marketing is not enough to prod the free market to supply the educational choice that charter school advocates feel is necessary, so pro-choice advocacy organizations also lobby state representatives and fund pro-charter board candidates.

Charter school marketing is extensive and wide-reaching \parencite{Finalsite2024, CSC2019, Cohen.Lizotte2015}. Organizations like The 74 Million, a reference to the 74 million children in America \parencite{The74Million2024} or Innovate Public Schools \parencite{InnovatePublicSchools2014} an advocacy organization, produce reports, news items, briefs and what claims to be research that is slanted toward charter schools and away from public schools. One example is an article, \citetitle{Aldeman2024} by \citeauthor{Aldeman2024} that references another article \parencite{WBUR2023} that is based on a unpublished, not peer reviewed study whose conclusions were based on self-reported data from ``168 responses to the survey (9\% of the potential pool), 84\% indicated hiring at least one ELH. This group of administrators was split on why they hired ELHs; half indicated that [ELHs]emergency license holders were the strongest applicants in the hiring pool, while the other half indicated that ELHs were the only applicants for the position'' \parencite[3]{Bacher-Hicks.etal2023}. No attempt was made to adjust for potential bias which the paper actually refers to: ``On the survey, school administrators (n=120) who had hired [emergency license holders] were most likely to say that they were just as effective or more effective relative to other newly hired teachers across all four Massachusetts Educator Evaluation performance standards'' \parencite[3]{Bacher-Hicks.etal2023}. 

These influence techniques are reminiscent of how OxyContin was marketed by the Sackler family, which is not surprising since Jonathan Sackler, now deceased, founded or funded charter advocacy groups like 50CAN, ConnCAN, Families for Excellent Schools, the Northeast Charter School Network, Education Reform Now, Partnership for Educational Justice, and The 74 Million. \textcite{Dubb2017} describes the similarities in marketing strategies used to sell oxycontin and those used to promote charter schools, where the focus of all communications was to highlight benefits while ignoring or erasing harms. While this is the standard playbook of corporate marketing, we now have public education dollars being spent on such tactics. When a national exposé published by National Public Radio (NPR) documented serious concerns about Rocketship's practices \parencite{Kamenetz2016}, The 74 Million immediately published an \textit{ad hominem} attack on NPR, accusing the report to have been a ``hit piece'' on the charter network \parencite{Smith2016a}. The response of The 74 Million addressed some of the issues raised by NPR while leaving unanswered some of the most serious concerns.

Unlike many other forms of privatization, charter schools have competition. When a local government turns over the task of supplying water to a town, for example, there is not another public water company serving the same customers to serve as a comparison. Privatization is often an all-or-nothing proposition. Charter schools, on the other hand, can be and are often compared to the public schools in the same school district. The presence of very visible competition has an interesting consequence: charter schools view public schools as an existential threat, precisely the opposite of the cooperative, synergistic relationship that state legislators envisioned. In fact, reports on the successful sharing of innovations appear so infrequently that sharing might as well be completely absent. 

Given that charter schools in California get the same per pupil funding as do public schools, there are a limited number of ways that charter schools can generate ``excess'' funds. They can lower operating costs by putting students in front of computers for 25\% of the day which allows them to hire fewer teachers. According to the Christensen Institute, this is worth \$500K per year per school in 2011\footnote{Assuming 3.5\% inflation for 13 years means \$500K in 2011 is the equivalent of \$782K in 2024.} \parencite{ChristensenInstitute2011}. They can tap into state or federal facilities grants unavailable to public schools like the Paycheck Protection Program \parencite[18]{RSEA2020}. \textcite{Baker.Miron2015} catalog the creative ways that charter schools across the United States operate to make money.

Charter schools employ fewer and less experienced teachers than public schools do. A teacher with 10 or 20 years of experience can easily command a salary that is twice that of a newly minted teacher. Rocketship schools have a student-to-teacher ratio that is officially as high as 35:1 \parencite[44]{SCCOE2023}, and if aides are counted as teachers, it is an estimate which understates the number of students per teacher. The combination of fewer and less expensive teachers can reduce the cost of teacher salaries to one-third of what public schools pay for teachers. This reduction is significant because teacher salaries typically account for from one-third to three-quarters of the total expense of running a school. Charter schools that employ a blended pedagogy can further reduce the cost of salaries, with virtual schools dispensing entirely with teachers, effectively reducing the single largest component of running a school to zero.

\subsubsection{Philanthrocapitalism}\label{sec:philanthrocapitalism}\indent

\index{philanthrocapitalism|(}
Philanthrocapitalism is the term used to describe the approach to philanthropy that prioritizes operating non-profits as businesses, i.e.~making money while ``doing good''. The epigraph to \citeauthor{Giridharadas2018}'s book \citetitle{Giridharadas2018} is a quote taken from Leo Tolstoy's \textit{Writings on Civil Disobedience and Nonviolence} which captures the absurdity of making money while ``doing good'':

\begin{quotation}\noindent
I sit on a man's back choking him and making him carry me, and yet assure myself and others that I am sorry for him and wish to lighten is load by all means possible \ldots\ except by getting off his back.
\end{quotation}

For philanthrocapitalists, the techniques and vehicles used to extract a profit from public education are extensive and often hidden from view. \textcite{Saltman2018c} lists the following in \citetitle{Saltman2018c} (pp.\textit{xii}–\textit{xiii}): social impact bonds, higher education lending and student income loans, charter school real estate, tax credit, and municipal schemes, and philanthrocapitalist educational technology schemes.
\textcite{Marachi.Carpenter2020}, \textcite{Burris.Cimarusti2021}, \textcite{Scott2009}, \textcite{Baker.Miron2015} make similar claims along the lines that education has been captured by big business, and where substantial, possibly hidden profits are to be made.
\index{philanthrocapitalism|)}


%%% Local Variables:
%%% mode: latex
%%% TeX-master: "Rocketship_Education-An_Exploratory_Public_Policy_Case_Study"
%%% End:
