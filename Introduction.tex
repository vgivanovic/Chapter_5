%%% Time-stamp: <2023-12-21 16:58:01 vladimir>
%%% Copyright (C) 2019-2023 Vladimir G. Ivanović
%%% Author: Vladimir G. Ivanović <vladimir@acm.org>
%%% ORCID: https://orcid.org/0000-0002-7802-7970

\begin{comment}
This section provides a general introduction to the area of study and presents the problem to be
investigated in the study. The purpose of the study needs to be clearly stated and describe the
following:
 a. The unresolved issue in education
 b. The significance of the problem
 c. The justification for investigating the problem
 d. An explanation of the importance of conducting a study to help resolve that issue
 e. Initial definitions for important terms and concepts likely to be used throughout the proposal
\end{comment}

\chapter{Introduction}\label{ch:Introduction}
\bigskip%
If, in Harold Lasswell's words, politics\index{politics!definition of} is about who gets what, when, and how \parencite{Lasswell1936}, then education is surely one of the most consequential — and fascinating — of public policy issues. At stake is the well-being of tens of million of students on whose behalf federal, state, and local governments spend upwards of three quarters of a trillion dollars annually.\footnote{The 50 states and the federal government spent \$734.9B in 2017–18. Using an inflation rate of 2\%, spending for 2021–22 would be just shy of \$800B. (Author's estimate using data from ``Revenues and Expenditures for Public Elementary and Secondary Education: FY 18'', NCES, 2020)} The number of stakeholders is huge: every parent and every child is a stakeholder, as are teachers, administrators, legislators, employees of fifty state departments of education, the federal Department of Education, the President of the United States, the U.S. Supreme Court, and state and local courts. Stakeholders exist throughout the United States, in states, counties, cities, towns, villages, and in almost 100 thousand schools in thousands of school districts. The COVID-19 pandemic of the last 2+ years has revealed just how important public education is.

Education is the arena in which parents, legislators, unions, political parties, billionaires, technologists, scholars and educators clash, all vying for influence and reward. Education is where religion, politics, free market neoliberalism, and social justice intersect. One topic in particular has, in the last fifty years, generated a disproportionate share of discord: the privatization of public education, i.e.~school choice.\index{school choice}\footnote{``School choice'' is an Orwellian name designed to mislead, to dress up an otherwise unpalatable reality: privatization takes something that used to be available to all and restricts it exclusively to those who can afford to pay.}

Formerly sleepy school board elections have attracted national interest, and with that interest, a flood of money. The 2020 Los Angeles school board election cost over \$14M for just four seats and generated articles in the national press. Likewise, a November 2016 statewide proposition in Massachusetts which sought to expand charter schools was covered extensively by national newspapers with one advocacy group spending more than \$15M (not including a \$425,000 fine for violating campaign law).\footnote{Details of the financing of the Great Schools Massachusetts 2016 ballot committee are spelled out in \textcite{Cunningham2021}.} Betsy DeVos, U.S. Secretary of Education under the former President Donald Trump, drew fierce criticism from the start of her tenure with her unwavering support of charter schools, criticism which was endlessly reported on. In short, charter schools became nationally visible. 

\section{Schools and Charter Schools}\indent

Most schools in the United States are either traditional public schools, charter schools, or private schools, with one catchall category: alternative schools.  Only two states, Nebraska and North Dakota, have resisted all forms of school choice; all states have private schools and an extensive public school system. By definition, school choice encompasses charter, private, magnet, and homeschooling, i.e.~every kind of school traditional except public schools. But, because school vouchers\index{school vouchers} in particular are becoming more common, school choice now increasingly refers to school vouchers in addition to charter schools \parencite{Enlow2022}.

Schools, under this definition of school choice, take a number of forms\index{schools!forms of}: they can, like traditional public schools be in-person, but unlike traditional public schools, they can also be completely online (virtual), or even a blend of virtual and in-person. How school choice is financed varies as well. School vouchers, various types of tax-credits, savings accounts, and tax deductions, have all been used, often augmented by tax dollars.\index{school choice!forms of financing} The phrase ``school choice'' is also associated with 529 savings accounts, student income loans, social impact bonds\index{social impact bonds}, and philanthrocapitalism\index{philanthrocapitalism}.\footnote{The use of a market-based approach in philanthropy.}

Regardless of how school choice is financed, school choice complicates what used to be a system of mostly public schools plus a few private schools that had been in place for over 150 years. This new kind of financing has raised some fundamental questions: Who benefits from this new financing? Do the children for whom education is the difference between being poor and ``flourishing benefit? Is education is being turned into a low-risk, profitable investment for hedge funds, private equity firms, investment banks, and the one percent?

The various forms of school choice have waxed and waned, but charter schools were present at the creation of the privatization movement in education\index{privatization!in education} and have continued to enroll more and more students, diverting more and more dollars out of the public school system \parencites{Lafer2017a}{Lafer2018}{Lafer.etal2021}.
\begin{comment}
  \parencites[131–132]{Lafer2017a}[18]{Lafer2018}[9]{Lafer.etal2021}
\end{comment}
School choice has spawned an entire industry devoted to marketing school choice: academic departments and institutions, educational associations, think tanks, astroturf\index{astroturf}\footnote{Wordnik definition: astroturf: ``The disguising of an orchestrated campaign as a ``grass-roots'' event – i.e., a spontaneous upwelling of public opinion.''} advocacy groups, and political action committees, all of which are examples of the marketing of the privatization of public education. %chktex 38

According to the National Center of Education Statistics in the U.S. Department of Education, there were 7,547 elementary and secondary charter schools\index{charter schools!number of U.S.} in the United States enrolling 3,431,230 students in 2019–20 school year \parencite[Table 216.90, p.144]{DeBrey.etal2022}. This represents 7.7\% of the total number of elementary and secondary schools and 6.8\% of the total number of students in the United States. The state with the greatest charter school presence was California which had 1,321 schools (12.7\% of the total) and 674,652 students (11.0\%). Within California, in the 2019–20 school year, charter schools\index{charter schools!enrollment of in Santa Clara County} in Santa Clara County enrolled 31,584 students (13.6\% out of 231,865) \parencite{CDEDataQuest2021}.

These are notable patterns, and the COVID-19 pandemic has accelerated the growth of charter schools, in contrast to the small decline of recent years. However, this recent growth appears to be almost completely due to the expansion of virtual charter schools\index{virtual charter schools!expansion of} \parencite{Strauss2021}. Despite continued growth, charter schools remain controversial and have generated heated debate. Reports and studies from charter school opponents have been answered by reports and studies from charter school advocates. Both sides claim their methodology to be superior and consider the other side's fatally flawed.\footnote{Jeffery Henig in his book \textit{Spin Cycle: How Research is Used in Policy Debates: The Case of Charter Schools} \parencite{Henig2009}, offers a detailed examination of the war of words that resulted from just one report and one newspaper article.}

What the research indicates – again and again – is that \textit{some} charter schools, under \textit{some} circumstances, for \textit{some} students, seem to do \textit{somewhat} better than traditional public schools.\index{charter schools!compared to public schools} \citefirstlastauthor{Garcia2018} notes that charter schools start out doing somewhat worse than public schools, but improve over time, with ``no discernible difference'' \parencite[119]{Garcia2018} after about five years of operation.

On the other hand, the Lubienskis showed after careful and thorough statistical analysis in \textcite{Lubienski.Lubienski2014} that public schools out perform charter schools.\index{charter schools!out performed by public schools} The Lubienskis used restricted-access 2003 NAEP data from just shy of 300,000 students in 4\textsuperscript{th} and 8\textsuperscript{th} grades in 6041 schools throughout the United States, plus data from the Early Childhood Longitudinal Study, Kindergarten (ECLS-K 98) class of 1998–99.\footnote{The Lubienskis were exceedingly thorough in their statistical analysis and devote over 80 pages in \textcite{Lubienski.Lubienski2014} to the details of their two-level hierarchical linear mode (three level for the ECLS-K 98 data). Their data is available from the National Center for Educational Statistics to qualified researchers, so that their analysis can be replicated.} So, based on the Lubienski’s analyses, there is no evidence that, on the whole, charter schools are superior to traditional public schools in academic performance. Rather, at best, they perform, on average, similarly.

If charter schools are on average no better than public schools, why are they so fervently touted as the answer to the perceived ills of American public education? Why are eye-popping sums (10× the usual amounts) spent supporting public school board candidates who favor charter schools? Why are charter schools still growing in both enrollment and in number? Is the profit motive is the overriding goal of charter schools, or are they instead driven by a genuine desire to improve the educational outcomes of the very children who could most benefit from a quality education? My goal in this dissertation is to offer some answers to questions like these by examining in detail the finances and financial structure of a single charter school chain, Rocketship Education, and entities associated with it.

I will use the term \textit{charter school chain}\index{charter school chain} to refer both to for-profit and non-profit organizations that manage more than one charter school since both take both financial and operational control away from schools and centralize it outside of schools, much like public schools are part of a public school district.  Charter school chains are essentially franchise operations like McDonald's or Hertz, but in education instead of hamburgers or rental cars. For-profit charter school chains have traditionally been called \textit{educational management organizations \index{educational management organization}(EMOs)}\index{EMO|see {educational management organization}} and non-profit charter school chains \textit{charter management organizations}\index{charter management organization}\index{CMO|see {charter management organization}}, but since there is little difference between the two, I will use \textit{charter school chains} when the distinction is unimportant.

The remainder of this chapter provides some context for why I conducted this study. The chapter \textit{\titleref{ch:litreview}} discusses the extensive literature on charter schools. The following chapter, \textit{\titleref{ch:methods}}, details what data will be collected, how it will be collected, and how it will be analyzed. The chapter~\textit{\titleref{ch:findings}} provides the results of analyzing that data in context of this study's research questions. The last chapter, \textit{\titleref{ch:discussion}} considers the limitations and public policy implications of my study and its conclusions. Finally, it makes some suggestions for how current public policy should be changed to achieve some of the seven goals that the California Legislature set out in \textit{The Charter School Act of 1992}.

\section{What is the Purpose of this Study?}\indent

The goal\index{dissertation!goal of} of this case study is to determine if Rocketship Education is, or might be, profitable and if so, how are these profits generated. It seeks to analyze as carefully and fully as possible the finances of Rocketship Education and of associated entities, concentrating on its real estate dealings.

Real estate, for charter schools, is of special significance because they have no facilities when they submit their initial petition. They do have several ways of obtaining the needed facilities, but because they cannot raise property or parcel taxes, nor can they pass a bond measure that is paid for by property taxes, charter schools must either obtain facilities from their home public school district or they must lease or buy facilities using funds they themselves have raised. Furthermore, since Rocketship Education is incorporated as not-for-profit corporations, any profits must remain as assets of Rocketship Education with one exception: profits may be transferred to another non-profits whose public benefit is similar to Rocketship's.

The non-real estate finances of charter schools\index{charter schools!finances of} — at least in California — are similar to public schools. Both use the same state mandated accounting structure because both have very similar needs. Although a charter school may pay more for this or less for that, fundamentally the revenues and expenses of charter schools are similar to those of traditional public schools. But when leasing, buying and potentially constructing facilities enter the picture, a different calculus ensues since significant sums are at stake. For example, a single transaction might be in the range of tens of millions of dollars.

This study concentrates on Rocketship Education\footnote{A note on names: Rocketship Public Schools is name that Rocketship Education is doing business as starting in June 2020, but since it has been known as Rocketship Education for much longer than it has been as Rocketship Public Schools\index{Rocketship Public Schools (fn)}, this study uses (mostly) the former name. Also, this study uses just Rocketship to refer to Rocketship Education and related entities, such as the various Launchpad Development LLCs\index{Launchpad LLCs (fn)} that are associated with individual schools.} because its popularity has led to core aspects of its model being adopted by other charter school chains such as the Caliber Public Schools\index{model!Rocketship!adopted by Caliber Public Schools (fn)} or the \index{model!Rocketship!adopted by Navigator Schools (fn)} Navigator Schools, both in California.  It is an exemplar of a popular charter school and has had an outsized influence on public education in Santa Clara County.

\index{model!Rocketship|(}This study seeks to determine if Rocketship Education or related entities are generators of profit. Furthermore, if the model that Rocketship Education uses does generate profits, can that model be used by other charter school operators within California or perhaps in other states? Many studies have examined the educational outcomes of charter schools and of charter chains, including one specifically on Rocketship's effect on Milwaukee's public schools had proposed legislation passed, but Rocketship's finances, with its real estate transactions as a focus, have not been studied in detail.\index{model!Rocketship|)}

It should be noted that this study will not examine the educational outcomes of Rocketship. All charter schools offer themselves as better alternatives to traditional public schools. Rocketship, for example, claims that its pedagogical model of blended learning
\begin{itemize}
  \item is more efficient than that of traditional public schools,
  \item offers personalized learning\footnote{Note that personalized learning is not the same differentiated instruction. All students follow the same path with personalized learning, albeit at different rates, instead of following different paths at different rates, as with properly implemented differentiated instruction.} through computer-mediated instruction, and
  \item yet still offers a human connection (at least part of the time) that is similar to traditional public schools.
\end{itemize}
These claims can and should be tested in other studies by comparing individual Rocketship schools to independent charter schools and to traditional public schools in the same district. The Rocketship chain may be compared to other charter school management organizations, to portfolios of charter schools, as well as to traditional public school districts, but such studies need to be done with care to avoid methodological errors that would reduce the validity of their conclusions.

\subsection{Research Question}\label{sec:research-questions}\indent

These questions and themes lead to the following research question: Has Rocketship structured itself to earn a return for its founders and investors, focusing especially on its real estate transactions? In order to answer this research question definitively, this study must be as complete as possible, and that entails understanding the finances of public schools in California, those of charter schools in California, and finally, those of Rocketship Education and related entities.\index{research question}

More broadly, there are additional reasons for studying charter school finances. Are we (the states, the federal government) misallocating the money we spend on charter schools? Could we be spending our tax dollars more wisely? What did taxpayers get for these expenditures? These questions, however interesting and appealing they may be, are beyond the scope of this study and remain for future researchers to explore.

\index{charter schools!finances!study of|(}This case study is unique in that it examines in depth the finances of a single charter school chain. There have been studies of the finances of aggregations of charter school chains (e.g..~all known charter school chains in the United States,\footnote{See \textcite{Miron.etal2021} for a list of currently known charter school chains.} or a selected group of charter school chains). Other studies have explored the effects of charter schools on segregation or academic achievement, or the financial impact of charter schools on their surrounding public school district. But academic studies of the finances of just a single charter school chain seem to be missing.\footnote{I distinguish between academic studies and criminal investigations. Clearly, the grand jury indictment of 11 persons associated with A3 Education was a study of a single charter school chain, but it was a criminal investigation, not an academic study.} Further, studies focusing on real estate of a single chain do not seem to have been performed. It is hoped that the lessons learned from this case study will be used by policy makers to strengthen charter school law in California and elsewhere in order to increase desired outcomes and to minimize cost and unintended consequences.

As tempting and as important as it might be, this dissertation will not examine the academic outcomes of Rocketship or of other charter schools. This dissertation will restrict itself to the finances of those schools. Much excellent work has already been done evaluating charter school outcomes. Section~\ref{sec:charter-surveys} discusses four surveys of charter school research and one overview book. \index{charter schools!finances!study of|)}

\section{Theoretical and Conceptual Frameworks}\indent

\index{dissertation!framework of|(}
According to \textcite{Grant.Osanloo2014}, creating and understanding the theoretical framework for one's dissertation is ``one of the most important aspects in the research process.'' (p.12) They liken the theoretical framework of a dissertation to the blueprints that define a house. That framework both defines the organization and the structure of a dissertation, as well as what counts as elements and their relationships. A theoretical framework articulates ``the researcher's understanding of how the research problem will best be explored, the specific direction the research will have to take, and the relationship between the different variables in the study'' \parencite[16–17]{Grant.Osanloo2014}. %chktex 38

Further, a ``conceptual framework offers a logical structure of connected concepts that help provide a picture or visual display of how ideas in a study relate to one another within the theoretical framework'' \parencite[16–17]{Grant.Osanloo2014}. This dissertation uses a case study approach as its conceptual framework within a public policy framework, its theoretical framework. 

\subsection{Public Policy as a Theoretical Framework}\label{subsec:PublicPolicyFramework}\indent
\index{framework!theoretical|(}
A public policy framework provides a rich set of tools and techniques with which to analyze Rocketship's finances. Three factors support using a public policy framework to guide understanding and evaluating Rocketship's finances. First, charter school finance is constrained primarily by public policies set by state legislatures, the creators of charter schools. These laws regulate taxes, grants, borrowing capacity, and reporting requirements of charter schools and charter school chains \parencite{Aguinaldo.etal2020}, and by definition, whatever falls within the purview of legislators is public policy. Second, \textcite{Brighouse.etal2018}, in \citetitle{Brighouse.etal2018}, provide a succinct definition of what public policy analysis is which matches the purpose of undertaking this case study. They use a values, evidence, and decision-making framework ``to make judgments about how well specific policies are likely to realize valued outcomes'' \parencite[p.1]{Brighouse.etal2018}. Last, these three concerns — values, evidence, decision-making — are considered the key concerns by academics and researchers in the public policy field \parencite{BuenoDeMesquita2016, Clemons.McBeth2021,Fowler2013,Gupta2011}. Using a public policy framework is appropriate when examining charter school finances.

The discipline of public policy sanctions a wide variety of tools and techniques when analyzing issues. (These tools and techniques will be discussed more fully in Chapter~\ref{ch:methods} or in Chapter~{ch:results} if and when they are used.) Public policy has been studied for years (there are public policy departments in many universities) and it is a mature area of academic research. As in most academic fields, there are fierce debates about the merits and robustness of a particular approach compared to alternatives, but at a high level, what to do is generally agreed upon. Most identify the following five steps (or variants thereof) that are used when creating public policy:

\begin{enumerate}
    \item Define the issues and set the agenda.
    \item Formulate one or more policies that address the issues identified.
    \item Evaluate those policies using tools and techniques like cost-benefit analysis, value analysis, political feasibility, game theory, and economic analysis.
    \item Implement those policies by passing legislation, changing practices, or by using the courts.
    \item Evaluate the effectiveness of the policy changes.
  \end{enumerate}
  
Two keys to identifying alternatives during policy formation and later when evaluating consequences are choosing or creating a model, and forecasting. Models identify what is going to be studied and their relationships, and forecasting is a prediction of the future whose consequences are (hopefully) identified in a model. \textcite{Page2018} lists 26 major models that have been used in science, business, and medicine. %chktex 12

The methodology of this dissertation draws on two excellent guides to public policy, \textcite{Clemons.McBeth2021} and \textcite{Gupta2011}. The first presents concepts, tools, and techniques used in analyzing public policy; the second a case study approach to public policy analysis. \textcite{Fowler2013} treats public policy in the field of education, but with an emphasis on power, politics, policy actors and the messy process of creating and implementing public policy.  \citeauthor{Clemons.McBeth2021} concentrate on explicating different theoretical approaches to public policy, whereas \citeauthor{Gupta2011} is the most practically oriented.

Since much of the evidence that will be presented will include financial data, the tools and techniques which manipulate and display data play an important role. First and foremost is statistical analysis. But, as \textcite{Epple.etal2016} show in Chapter~\ref{ch:litreview}, being clear on what exactly is being analyzed and what are the inherent limitations of that data is fundamental. It makes no sense to analyze brilliantly the wrong data or to stretch the data beyond its 
\index{framework!theoretical|)}

\subsection{A Case Study Approach as a Conceptual Framework}\indent
\index{framework!conceptual|(}

Broadly, social science research falls into one of two categories. The research may make many observations with a narrow focus, or may instead adopt a broader focus, but with a correspondingly smaller number of observations.  \citeauthor{Gerring2017} calls these ``large C'' or ``small C'' studies, respectively \parencite[xvii]{Gerring2017}. Of course, the boundary between large C and small C studies is not sharply defined.

Gerring calls small C studies \textit{case studies}. In this dissertation I study only one entity, Rocketship Education, and only one aspect of that entity, namely Rocketship's finances. But I consider the topic of  Rocketship's finances look at its finances broadly, examining as many different kinds of financial transactions as are publicly available for the subset of Rocketship schools that are in Santa Clara County. I discuss the elements of what makes a case study a good case study in Chapter~\ref{ch:discussion}, \textit{{\titleref{ch:discussion}}}.

\textcite{McCombes2019} says that case studies are a ``detailed study of a specific subject, such as a person, group, place, event, organization, or phenomenon''. They are `good for describing, comparing, evaluating and understanding kdifferent aspects of a research problem'' and are ``an appropriate research design when it allows you to explore the key characteristics, meanings, and implications of the case.'' Two papers go into detail about using the case study approach: \textcite{Crowe.etal2011,Rashid.etal2019}. \textcite{Yin2018} provides a detailed methodology for doing case study research well. %chktex 38

A case study framework for public policy research is ideal because the theory and practice of case studies is well-known and has been used both for public policy research and in  public policy analysis for years. A case study framework formalizes an in-depth examination of a single topic, in this case, the finances of Rocketship Education and related entities.
\index{framework!conceptual|)}
\index{dissertation!framework of|)}

This introduction has made the case that public education is important to many stakeholders, but that there is also discord around larger issues like values, ideology, and implementation. Charter schools have been offered as way of disrupting American public education from its hide-bound, archaic, and sclerotic present, driving it, despite opposition, into a dynamic future where education is tailored to each child's real needs. Establishing whether financial gain plays a key or even a primary role in American educational reform by carefully examining Rocketship's finances is both timely and important: Rocketship Education is growing, and with it, Launchpad Development. They have served as a model for other charter school chains in the United States.

%%% Local Variables:
%%% mode: latex
%%% TeX-master: "Rocketship_Education-An_Exploratory_Public_Policy_Case_Study"
%%% End: