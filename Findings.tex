%%% Time-stamp: <2024-01-16 18:23:42 vladimir>
%%% Copyright (C) 2019-2024 Vladimir G. Ivanović
%%% Author: Vladimir G. Ivanović <vladimir@acm.org>
%%% ORCID: https://orcid.org/0000-0002-7802-7970

\chapter{Findings}\label{ch:findings}%
This chapter presents the results of investigating Rocketship's finances using the approach outlined in \prettyref{ch:methods} whose goal was answering this dissertation's research question:
\begin{displayquote}
{Has Rocketship structured itself and its finances, to earn a return to investors, focusing especially on real estate transactions, and if so, how}?
\end{displayquote}

The first section presents Rocketship's corporate structure, a structure that separates Rocketship schools from Rocketship facilities. The next section, \prettyref{sec:location-and-property-info}, details what facilities Rocketship has in Santa Clara County, where those facilities are located, when they were acquired, and what real estate rights Rocketship has in those properties. Then, given Rocketship's real estate holdings, the third section characterizes the finances of Rocketship that were used to fund the purchase of those properties. The penultimate section reviews what gaps, anomalies and discrepancies were found in Rocketship's financial data. The final section, \prettyref{sec:issues_equality_equity}, looks briefly at issues of fairness.

Note that Rocketship financial data is not available for all years, and starting in 2014, annual financial statements are for all of Rocketship Education, i.e.\ schools in California, Tennessee, Wisconsin, Washington, D.C., and Texas.

\section{Rocketship's Corporate Structure}%
\label{sec:RSED-corporate-structure}\indent%

\index{Rocketship!corporate structure|(}
One of the original four members of Rocketship's board of directors was Eric Resnick, a specialist in real estate finance. He was expected to ``provid[e] a deep understanding of financial management and real estate transactions'' \parencite[13]{Danner2006}, so it appears that Rocketship's corporate structure was designed with real estate transactions in mind. From the start, Rocketship has kept schools and their facilities separate. This structure is diagrammed in \prettyref{fig:corporate-structure} on p.\pageref{fig:corporate-structure} for Rocketship facilities in Santa Clara County.

\begin{figure}[htbp]
  \centering\scriptsize
  \caption[Rocketship's Corporate Structure for Santa Clara County Facilities]{\emph{Rocketship's Corporate Structure for Santa Clara County Facilities}}\label{fig:corporate-structure}
  \sffamily
  \begin{forest}
    for tree={grow'=east, folder, draw, align=left}
    [ \textbf{Rocketship Education (RSEA)}, baseline
    [ Launchpad Development Company (LDC), xshift=2em
    [ \textit{Launchpad (LP)}, xshift=4em ]
    [ \textit{Launchpad Development One LLC (LLC1) RMS facilities}, xshift=4em ]
    [ \textit{Launchpad Development Two LLC (LLC2) RSSP facilities}, xshift=4em ]
    [ \textit{Launchpad Development Threee LLC (LLC3) RLS facilities}, xshift=4em ]
    [ \textit{Launchpad Development Four LLC (LLC4) ROMO facilities}, xshift=4em ]
    [ \textit{Launchpad Development Five LLC (LLC5) RDP facilities}, xshift=4em ]
    [ \textit{Launchpad Development Eight LLC (LLC8) RSA facilities}, xshift=4em ]
    [ \textit{Launchpad Development Ten LLC (LLC10) RSK facilities development}, xshift=4em ]
    [ \textit{Launchpad Development Eleven LLC (LLC11) RBM facilities}, xshift=4em ]
    [ \textit{Launchpad Development Twelve LLC (LLC12) RFZ facilities}, xshift=4em ]
    [ \textit{Launchpad Development Sixteen LLC (LLC16) RRS facilities}, xshift=4em ]
    ]
    [ Rocketship Support Network (RSN), xshift=2em ]
    [ Rocketship Mateo Sheedy Elementary (RSM), xshift=2em ]
    [ Rocketship Sí-Se-Puede Academy (RSSP), xshift=2em ]
    [ Rocketship Los Sue (RLS), xshift=2em ]
    [ Rocketship Mosaic Elementary (ROMO), xshift=2em ]
    [ Rocketship Discovery Prep (RDP), xshift=2em ]
    [ Rocketship Alma Academy (RSA), xshift=2em ]
    [ Rocketship Brilliant Minds (RBM), xshift=2em ]
    [ Rocketship Spark Academy (RSK), xshift=2em ]
    [ Rocketship Fuerza (RFZ), xshift=2em ]
    [ Rocketship Rising Stars (RRS), xshift=2em ]
    ]
  \end{forest}
\end{figure}
\index{Rocketship!corporate structure|)}

\index{Rocketship!corporate structure!ownership details|(}
The parent corporation, Rocketship Education, Inc. (RSED) was formed in California on February 16, 2006  as a 501(c)(3) public benefit corporation under California's Nonprofit Public Benefit Corporation Law (California Corporation Code §§5000–10841). RSED owns all the Rocketship schools and Launchpad Development Company, a 509(a)(3) nonprofit public benefit corporation.\footnote{A 509(a)(3) corporation is a ``charity that carries out its exempt purposes by supporting other exempt organizations, usually other public charities'' and ``has a relationship with its supported organization sufficient to ensure that the supported organization is effectively supervising or paying particular attention to the operations of the supporting organization.'' \parencite[accessed 29 Sep 2023]{IRS2023}}. RSED plus the schools plus Launchpad Development Company are known collectively as Rocketship Education and Its Affiliates (RSEA). RSEA is not a recognized corporate entity. It's just a convenient way of referring to all things Rocketship.
\index{Rocketship!corporate structure!ownership details|)}

\index{Rocketship!corporate structure!Launchpad Development|(}
Launchpad Development Company owns one nonprofit public benefit corporation LLC for each school, generally named ``Launchpad Development <number> LLC'', and that entity owns the actual school facilities. In addition, Rocketship has two functional divisions:
\begin{itemize}
  \item Rocketship Support Network (RSN) which provides resources for management, back-office support, and organizational strategy to Rocketship schools, and
  \item Launchpad (LP) which provides investment and asset management, and administrative services to Launchpad LLCs.
\end{itemize}
\index{Rocketship!corporate structure!Launchpad Development|)}

This separation between the operation of schools from the funding of their facilities raises the question of why Rocketship has chosen this structure. \citefirstlastauthor{Reuting2023} in \citetitle{Reuting2023} is unequivocal, ``LLCs are the best entities for holding real estate, no doubt about it. They offer the most liability protection of any entity type out there, and when you're looking to protect valuable assets, this peace of mind is priceless.'' \parencite[292]{Reuting2023}

\index{Rocketship!corporate structure!rationale|(}
Rocketship itself gave four reasons for this corporate organization:
\begin{enumerate}
  \item The need to eliminate RSED liability. Without Launchpad and its LLC's, RSED is taking on several liabilities
  \begin{itemize}
    \item developing financing deals
    \item lawsuit related to CEQA
    \item financial risk from financing
  \end{itemize}
  With Launchpad and LLCs, RSED will have no liability associated with real estate.
  \item The need to manage RSED's cash flow which fluctuates whenever a new school is financed. These fluctuation are large and lead to unnecessary speculation.\footnote{What that speculation is, was not specified.}
  \item The need to allow RSED to focus on ``Great Schools'' and to let Launchpad focus on building ``Great Sites''.
  \item The need to increase the market for developers of charter facilities.
\end{enumerate}
At a Board offsite on 23 Jun 2009, Rocketship expanded significantly on these reasons in an 18 page presentation, \textcite{RSED2009}.
\index{Rocketship!corporate structure!rationale|)}

\section{Rocketship Locations and Property Information}
\label{sec:location-and-property-info}\indent%

\index{Rocketship!facilities!requirements for|(}
Before the formation of Launchpad, the Rocketship board chose sites for its schools according the following criteria:
\begin{itemize}
  \item Location: Within 1 mile of a PI\footnote{A PI is a school under Program Improvement, a designation under Title I, that indicates that the school has failed to show adequate yearly progress for two consecutive years and is therefore subject to improvement or corrective action measures} or otherwise low performing school, 
  \item Qualifies under the New Market Tax Credit Criteria (75\% FRL).
\end{itemize}
In September 2009, they added
\begin{itemize}
  \item Financials: Less than \$8M for 30 years at 5\% interest.
  \item Enrollment: For a school with 450 K-5 students, at least 3× that number within 1 mile or compensating interest from families outside the 1 mile radius.
\end{itemize}
These selection criteria appeared very early on in September 2009\parencite{RSED2009b} and they demonstrate that Rocketship was aware of the NMTC criteria and choose schools which would qualify.
\index{Rocketship!facilities!requirements for|)}

Details of the Rocketship schools listed in \prettyref{tab:locations} are given in \prettyref{appx:rocketship-property-info} on p.\pageref{appx:rocketship-property-info}.

\index{Rocketship!properties|(}
\begin{table}[hbt]
  \caption[Rocketship Property Information]{\textit{Rocketship Property Information}}%
  \label{tab:locations}\SingleSpacing%
  \begin{tabular}{lll}
    \toprule
    School          & Address                               & Property Information \\
    \midrule
    Mateo Sheedy    & 788 Locust St., San José, CA 95110    & \prettyref{sec:mateo-sheedy-info}%
     \index{Rocketship!schools in Santa Clara County!Mateo Sheedy} \\
    Sí Se Puede     & 2249 Dobern Ave, San José, CA 95116   & \prettyref{sec:sí-se-puede-info}%
     \index{Rocketship!schools in Santa Clara County!Sí Se Puede} \\
    Los Sueños      & 331 S. 34th St, San José, CA 95116    & \prettyref{sec:los-suenos-info}%
     \index{Rocketship!schools in Santa Clara County!Los Sueños} \\
    Discovery Prep  & 370 Wooster Ave, San José, CA 95116   & \prettyref{sec:discover-prep-info}%
     \index{Rocketship!schools in Santa Clara County!Discovery Prep} \\
    Mosaic          & 950 Owsley Ave, San José, CA 95122    & \prettyref{sec:mosaic-info} %
     \index{Rocketship!schools in Santa Clara County!Mosaic}\\
    Brilliant Minds & 2960 Story Rd, San José, CA 95127     & \prettyref{sec:brilliant-minds-info} %
     \index{Rocketship!schools in Santa Clara County!Brilliant Minds}\\
    Alma Academy    & 198 West Alma Ave, San José, CA 95110 & \prettyref{sec:alma-academy-info} %
     \index{Rocketship!schools in Santa Clara County!Alma Academy}\\
    Spark Academy   & 683 Sylvandale Ave San José, CA 95111 & \prettyref{sec:spark-academy-info} %
     \index{Rocketship!schools in Santa Clara County!Spark Academy}\\
    Fuerza          & 70 S. Jackson Ave, San José, CA 95116 & \prettyref{sec:fuerza-info} %
     \index{Rocketship!schools in Santa Clara County!Fuerza}\\
    Rising Stars    & 3173 Senter Road, San José, CA 95111  & \prettyref{sec:rising-stars-info} %
     \index{Rocketship!schools in Santa Clara County!Rising Stars}\\
    \bottomrule
  \end{tabular}
\end{table}
\index{Rocketship!properties|)}

\section{Rocketship's Finances}
\label{sec:rocketship_finance}\indent%

\subsection{Rocketship Financial Documents}%
\label{sec:rocketship-financial-docs}\indent%

Every year, as required by law, Rocketship issues an independently audited financial statement for the preceding school year. Rocketship, rather than issuing a separate financial statement for each of its affiliates, consolidates them into a single document, typically called \textit{Rocketship Education, Inc.\ and Its Affiliates Consolidated Financial Statements and Supplementary Information Year Ended June 30, <year>}. Four annual financial statements are reported:

\begin{itemize}
  \item Financial Position, which corresponds to a business's balance sheet\index{Rocketship!financial statements!financial position}
  \item Activities, which corresponds to a business's income statement\index{Rocketship!financial statements!activities}
  \item Cash Flows, which corresponds to a business's cash flow statement\index{Rocketship!financial statements!cash flows}
  \item Functional Expenses, which is usually only used by non-profits\index{Rocketship!financial statements!functional expenses}
\end{itemize}

\index{Rocketship!financial statements|(}
The four different financial statements for the years 2010–2022\footnote{The years ending 2006–2008 were rolled up as single year that was not reported on until 2009. That year included all previous years and was restated in 2022. That makes the year 2009-2010 the first ``normal'' year} have been collected and the data summarized in \prettyref{tab:consolidated_financial_position}, \prettyref{tab:consolidated_activities}, \prettyref{tab:consolidated_cash_flows}, and \prettyref{tab:consolidated_functional_expenses}, in Appendices D – G, and online in this dissertation's \textit{Data Dashboard}, a Google spreadsheet.\footnote{\url{https://docs.google.com/spreadsheets/d/1c4akEKFj9bmVfLFQwi7ewMifSjRbrw5xpjh_UjO4oYY/edit?usp=sharing}}
\index{Rocketship!financial statements|)}

\index{Rocketship!financing charter schools|(}
These four annual financial statements allow a detailed view of how Rocketship finances its facilities. Financing charter schools in California is more complicated than the financing of traditional public schools because charters need to obtain facilities often independent from the public school district in which they are located.
\prettyref{tab:charter-school-financing} on p.\pageref{tab:charter-school-financing} describes what facility financing options a charter school has compared to a traditional public school. Note that ending up with facilities that satisfy a school's needs may require the purchase of land, the construction of new facilities, or the modification of existing facilities in addition to operating those facilities. Each of these alternatives may require different financing methods.

To illustrate the variety of financing options that may be used, Rocketship states that they used three different financing options for nine schools as of 2015.
\begin{displayquote}
Launchpad successfully financed four of the nine Bay Area Rocketship projects with New Market Tax Credits, four projects by issuing long term tax exempt bonds, and one project through short term private financing.
\end{displayquote}
\index{Rocketship!financing charter schools|)}

\index{Rocketship!financial model and forecast|(}
Rocketship also prepared a detailed spreadsheet of their financial model, \textit{Current RSED Financial Model 061909} \parencite{RSED2009a}, that extrapolates some data to 2045.

This spreadsheet contains a huge amount of data. Although the mass of documents asked for by FCMAT (Fiscal Crisis and Management Team) is much larger, the depth of analysis by Rocketship is significantly deeper. The entire spreadsheet is available at \url{https://docs.google.com/spreadsheets/d/1e5j8nn2Ofg6l5BlOaPi_qcByGH_OAt232RrvTkoJy2Q}. Rocketship maintains an archive of its board agends, materials, and minutes at \url{https://www.rocketshipschools.org/about/board-of-directors/board-agendas-archive/}, but it has removed agendas, meeting materials, and minutes prior to 2017. Any meeting material used here prior to that date was collected before Rocketship removed those files.
\index{Rocketship!financial model and forecast|)}

\index{Rocketship!financial model!school parameters|(}
Also revealing of Rocketship's early financial thinking is \citetitle{FinNarr2010} \parencite{FinNarr2010} (The first author's last name is not specified). In it, Rocketship describes the parameters of a typical school (see the sheet ``RSGen'' in \textcite{RSED2009a}) from one year before opening to year 10. Some noteworthy observations:
\begin{itemize}
  \item Rocketship expects each school to reach breakeven in year 1 of operation. \parencite[1]{FinNarr2010}
  \item Rocketship predicts that student demographics in years 1–8+ will be 70\% Free and Reduced Lunch with 50\% below the federal poverty level. English Language Learners (ELL) will drop from 70\% in year 1 to 50\% in year 8+ \parencite[1]{FinNarr2010}.
  \item The first seven schools are to pay 25\% of revenue (less food service sales and reimbursements) in management fees in the year before opening (year -1), dropping to 15\% in years 0–3+. Facilities fees start in year 1 and are 20\% of revenue (less food service sales and reimbursements). For schools 8+, there are no fees in years -1 through 1
  \item Rocketship expects each school to receive \$1.1M total in grants in its early years.
\end{itemize}
\index{Rocketship!financial model!school parameters|(}

\index{Internal Revenue Service Form 990|(}
In addition to the annual consolidated financial statements, Rocketship Education and Launchpad Development must file an IRS Form 990, a form that tax-exempt entities must file annually. This form is  public information which provides a different point of view than the annual financial statements because they concentrate (mostly) on what a nonprofit did rather than accounting for revenues and expenses.

Form 990, according to \citefirstlastauthor{Fishman2022} in \citetitle{Fishman2022} is intended to ``\ldots{} provide a complete picture of your nonprofit—including its activities, finances, governance, compensation, and tax compliance.'' \parencite[84]{Fishman2022} In addition to Form 990 itself (12 pages) there are potentially 16 additional schedules for a total of 80 pages. Instructions run 300 pages. Rocketship Education and Launchpad Development Company Form 990s are available at \url{https://apps.irs.gov/app/eos/} and on the GuideStar web site.\index{Internal Revenue Service Form 990|)}

\subsection{Financing of Charter Schools}%
\label{sec:financing_charter_schools}\indent%

\index{charter schools!facilities!financing of|(}
In addition to needing revenue to finance operations, charter schools will likely need money to finance the purchase or lease of facilities if they don't avail themselves of district facilities via Prop. 39. \prettyref{tab:charter-school-financing} below lists the major sources of revenue that fund both operations and facilities.

\begin{table}[ht]
    \OnehalfSpacing%
    \caption[Charter School Financing]{\textit{Charter School Financing}}\label{tab:charter-school-financing}%
    \begin{tabularx}{\textwidth}{llccl}
      \toprule
      \textbf{Type}        & \multicolumn{2}{c}{\textbf{Available to}}  & \textbf{Notes}\\
                           & \textbf{TSPs} & \textbf{Charters}          & \\
    \midrule
    \multicolumn{5}{l}{\textit{State funding}}  \\
    \midrule
      \ding{51} & LCFF                 & Yes  & Yes                        & State minimum guarantee\\ 
      \ding{51} & Local property tax   & Yes  & Yes                        & Reduces LCFF amount\\
                & Categorical programs & Yes  & Yes                        & \multirow[t]{2}{2.5in}{All state funding outside of LCFF is\\
    categorical. Some federal programs exist.} \\
    \\
    \multicolumn{5}{l}{\textit{Local funding}}\\
    \midrule
      \ding{51} & Local parcel tax     & Yes  & No                         & district-wide election\\
      \ding{51} & Bonds                & Yes  & Yes                        & \multirow[t]{2}{3in}{Public schools: district election \\
    Charters: private or gov't sponsored}\\
    \\
    \multicolumn{5}{l}{\textit{Federal, state, or private funding}}\\
    \midrule
    \ding{51} & Private grants       & Yes & Yes                         & Much more common with charters\\
    \ding{51} & Venture fund loans   & No  & Yes                         & May use New Market Tax Credits\\
    \ding{51} & Rent subsidies       & No  & Yes                         & By the state (SB740)\\
    & COVID-19 PPP loans   & No  & Yes                         & Paycheck Protection Program\\
    \bottomrule
  \end{tabularx}
\end{table}\index{LCFF}

The first four sources of financing listed in \prettyref{tab:charter-school-financing} are considered ordinary revenue which are available to both public and charter schools, although the amounts and timing of the distributions vary. The remaining four are not necessarily present for a given charter school or public school district and typically are not used to fund operations. Entries with a checkmark are explained below.
\index{charter schools!facilities!financing of|)}

\subsubsection{LCFF}%
\label{sec:lcff}\indent%

\index{LCFF|(}
The Local Control Funding Formula is the principal way California funds both charter schools and public schools. County Offices of Education receive from the California Department of Education funds calculated using the Local Control Funding Formula (with adjustments) and those funds are distributed to all public school districts in the county. In turn, public school districts 
pass through an amount to charter schools in their district, also calculated using the LCFF.

All schools have the same base grant which varies by grade span. If a school, charter or public, has students who are in one or more of the following categories (1) eligible for free or reduced price meals (FRPM), (2) are English Learners (EL), or (3) are foster youth, the school receives a supplemental grant of 20\% of its base grant for each such student. If the qualifying population of students\footnote{These are unartfully called ``unduplicated pupils'' because schools do not get extra money for students in more than one category, as perhaps they should.} exceeds 55\% of the total number of students, a school receives a concentration grant of 65\% of the base grant for every student above the 55\% threshold. All Rocketship schools are located in high-poverty areas and all have more than 55\% of their students in at least one of the three categories that qualify for concentration grants. In sum, Rocketship gets LCFF base grants, supplemental grants, and concentration grants as part of it total LCFF funding.

Rocketship reports LCFF money as ``LCFF State Aid \& Property Tax Revenue'' or ``Apportionment revenue'' in its Consolidated Statement of Activities for the years 2010-2022 in this dissertation's Data Dashboard.\footnote{\url{https://docs.google.com/spreadsheets/d/1bnBIUkx7EPZU2UEUxi5M4BwkSgVjmKYVaZTnBZgIq8I} in the ``Consolidated Activities'' sheet.} Note that the definitions that Rocketship uses for these two categories of state revenue  are not comparable because California changed funding models from the Revenue Limit system (1973–1974 to 2012–2013) to LCFF (2013–2014 to the present).
\index{LCFF|)}

\subsubsection{Local Property Taxes}%
\label{sec:property-taxes}\indent%

\index{California!property taxes|(}
In California, commercial and privately owned properties are taxed, unless exempt, typically because they are religious organizations or are charities. School districts receive about 40\% of the property tax collected from properties in their home district and this tax replaces an equal portion of LCFF revenue. (If a district's property tax revenue exceeds what they would have gotten in LCFF funding, they receive no LCFF funding. These districts are called \textit{community-funded districts}, previously known as \textit{basic aid districts}.) Note that the amount that districts pass through from their LCFF allotment to charter schools is independent of how much property tax is collected; it is always the full LCFF amount. The Data Dashboard sheet ``Consolidated Activities'' lists the property tax revenue that Rocketship received from each school's home district.
\index{California!property taxes|)}

\subsubsection{Local Parcel Taxes}%
\label{sec:parcel-taxes}\indent%

\index{California!parceltaxes|(}
Traditional public school district may assess a non-\textit{ad valorem} tax, usually a per parcel tax\footnote{A 2023 court decision allowed a tax based on square footage because it is also a non-\textit{ad valorem} tax.} if voters approve. Charter schools do not have taxing authority, so they may not assess parcel taxes. However, Public school districts may agree to share some portion of their parcel tax revenue with charter schools within their boundaries, but are not required to do so. Rocketship has no parcel tax revenue.
\index{California!parceltaxes|)}

\subsubsection{Bonds}%
\label{sec:bonds}\indent%

\index{bonds!types of|(}
Bonds, as far as educational institutions are concerned, come in just a few forms\footnote{A complete discussion of the various forms, constraints on, and capabilities of governmental debt in California can be found in \citetitle{CDIAC2023} \parencite{CDIAC2023}}. These are:

\begin{itemize}
  \item General Obligation (GO) bonds are backed by the full faith and credit of the issuer, here a public school district or a charter school. Normally, bonds are secured by assets owned by the borrower, such as real estate, personal property (e.g.\ an airplane or an oil well), or some other physical asset. Lenders (the purchasers of a bond) are naturally reluctant to lend based on an ephemeral asset like a revenue stream because of the chance that the revenue stream might dry up. The solution for charter schools is conduit borrowing described below.

  Unlike public school districts that can pass a bond measure based on the value of the entire district's assessed property, charter schools have either no real property (if they are leasing) or a very small amount (if they own their facilities), so even if they were allowed to put a bond measure to the voters, the GO debt limit of 1¼\% of their facility's assessed value would provide very limited funds. For example, an \$80M valuation would be required to be able to issue a \$1M bond. 

  \item Tax and Revenue Anticipation Notes (TRANs) and Revenue Anticipation Notes (RANs) are backed by specific forms of revenue. These bonds are normally issued to smooth out state and local revenue streams and are of short duration.
  
  \item Conduit Bonds are a type of tax-exempt municipal bond issued by  a government agency (the conduit lender) that is neither the borrower nor the purchaser. The government entity or agency functions merely as a conduit between a borrower (often called the conduit borrower) and the purchaser of the bond (i.e.\ the lender). The conduit lender administers an offering by loaning to the borrower money it has received from issuing the bond. Often the funds come from a government agency (e.g. California Municipal Finance Authority, California School Finance Authority). The bond is an obligation of the borrower, not the conduit lender. The borrower's payments to the conduit lender are sized to meet the payments that the conduit needs to repay the purchaser(s) of the debt. The IRS publication \citetitle{IRS2019} details has some helpful information on conduit bonds and tax compliance.

\begin{comment}
  For example, in 2014, the California Municipal Finance Authority issued bonds (2014A:tax-exempt and 2014B:taxable) for \$32,915,000. These bonds were bought by Approved Institutional Buyers, and the proceeds were loaned to Launchpad Development Company which used them for public benefit. The borrower is Launchpad Development Company, and the conduit borrower is the California Municipal Finance Authority.
\end{comment}
\end{itemize}
\index{bonds!types of|)}

\index{Rocketship!bonds issued|(}
This dissertation's Data Dashboard shows that Rocketship issued bonds totaling over \$200M from 2011 to 2022. In the years ending 2008 through 2011, Rocketship borrowed at least fifteen times before actually issuing a bond. \prettyref{tab:rocketship_bonds} on p.\pageref{tab:rocketship_bonds} lists all the bonds issued by Rocketship or Launchpad from 2011 through 2022.

\begin{table}[ht]
  \SingleSpacing
  \caption{Rocketship Bonds}\label{tab:rocketship_bonds}%
  \begin{tabular}{llll}
    \toprule
    Name             & Amount    & Interest Rate  & Due Date       \\
    \midrule
    Series 2011A     & \$9.600M  & 8.5\%–9\%      & Dec 2041       \\
    Series 2011B     & \$0.515M  & 8.5\%–9\%      & Dec 2018       \\
                                                                   \\
    Series 2012A     & \$9.105M  & 6.25\%         & Jun 2042       \\
    Series 2012B     & \$0.355M  & 8.5\%          & Jun 2016       \\
                                                                   \\
    Series 2014A     & \$31.935M & 6.00\%–7.25\%  & Jun 2018,24,35 \\
    Series 2014B     & \$0.920M  & 6.00\%–7.25\%  & Jun 2016       \\
                                                                   \\
    Series 2015A     & \$6.135M  & 4.25\%         & Mar 2028       \\
    Series 2015B     & \$0.250M  & 4.25\%         & Jun 2016       \\
                                                                   \\
    Series 2016A     & \$28.080M & 4.25\%         & Mar 2046       \\
    Series 2016B     & \$0.525M  & 4.25\%         & Jun 2016       \\
                                                                   \\
    Series 2017A     & \$23.098M & 4.50\%–6.25\%  & 2027–2052      \\
    Series 2017B     & \$3.665M  & 4.50\%–6.25\%  & 2025           \\
    Series 2017C     & \$7.160M  & 4.50\%–6.25\%  & 2040           \\
    Series 2017D     & \$0.250M  & 4.50\%–6.25\%  & 2019           \\
    Series 2017E     & \$7.740M  & 4.50\%–6.25\%  & 2047–2052      \\
    Series 2017F     & \$0.250M  & 4.50\%–6.25\%  & 2019           \\
                                                                   \\
    Series 2017G     & \$15.560M & 4.05\%–6.00\%  & 2025–2053      \\
    Series 2017H     & \$0.665M  &  4.05\%–6.00\% & 2022–2025      \\
                                                                   \\
    Series 2019A     & \$28.075M & 5.0\%–5.3\%    & 2029-2056      \\
    Series 2019B     & \$0.935M  & 5.0\%–5.3\%    & 2020-2023      \\
                                                                   \\
    Series OG2021A   & \$14.780M & 4.0\%          & 2022–2035      \\
    Series OG2021B   & \$0.465M  & 4.0\%          & Jun 2022       \\
                                                                   \\
    Series OG2022A,B & \$27.990M & 4\%–4.5\%      & Jun 2022–2042  \\
    \midrule
    Total 2010-2011  & \$218.053M                                  \\
    \bottomrule
  \end{tabular}
\end{table}
\index{Rocketship!bonds issued|)}

Lines 64-86 of this dissertation's Data Dashboard (\url{https://docs.google.com/spreadsheets/d/1bnBIUkx7EPZU2UEUxi5M4BwkSgVjmKYVaZTnBZgIq8I}) show that as of June 30, 2022, the total that Rocketship Education and Affiliates owed was \$186,550,566, or just over \$32K per child.

\subsubsection{Private Grants}%
\label{sec:private-grants}\indent%

\index{Rocketship!private grants|(}
Rocketship lists a total of \$78,387,835 as ``Contributions'' from 2010 through 2022 (see line 11 of \prettyref{tab:consolidated_activities} in Appended E on p.\pageref{tab:consolidated_activities}). Unfortunately, the details of what those contributions actually consist of are mostly not available. It is difficult to tease out what is an unrestricted grant, what is a restricted grant, what is a forgiven loan, and what has been rolled over or consolidated given what data is publicly available. But, what is known is that Reed Hastings and the Walton Family Foundation each promised \$250K per school for schools two through seven. An additional \$600K, again per school, of federal funding through Title V was available. Each school thus gets \$1.1M in startup grants.
\index{Rocketship!private grants|(}

\subsubsection{Venture Capital Funding}%
\label{sec:venture-capital_funding}\indent%

\index{Rocketship!private grants|(}
Rocketship has made heavy use of loans and grants from venture capital funds. Most of these loans were forgiven, in one way or another, turning them into grants. To be forgiven, some benchmarks need have been met and almost always were.

\begin{table}[ht]
  \SingleSpacing
  \caption{Venture Capital Funding}%
  \label{tab:venture_captial_funding}
  \begin{tabular}{llll}
    \toprule
    Year & Name                           & Amount   & Interest Rate \\
    \midrule
    2010 & Charter School Growth Fund     & \$3.400M & 3.25\%        \\
    2012 & Charter School Growth Fund     & \$1.000M & 4.00\%        \\
    2013 & Charter School Growth Fund     & \$0.125M & 1.00\%        \\
    2013 & Charter School Growth Fund     & \$0.500M & 1.00\%        \\
    2013 & CSGF Revolving Facilities Loan & \$0.125M & 1.00\%        \\
    2014 & Charter School Growth Fund     & \$0.500M & 1.00\%        \\
    2014 & CSGF Revolving Facilities Loan & \$7.000M & 3.75\%        \\
    2016 & Charter School Growth Fund     & \$0.300M & 1.00\%        \\
    2016 & CSGF Revolving Facilities Loan & \$2.700M & 3.75\%        \\
    2017 & Charter School Growth Fund     & \$1.000M & 1.00\%        \\
    2019 & Charter Impact Fund            & \$7.300M & 4.40\%        \\
    \midrule
         & Total                          & \$23.95M &               \\
    \bottomrule
  \end{tabular}
\end{table}
\index{Rocketship!private grants|)}

\subsubsection{Rent Subsidies}%
\label{sec:rent-subsidies}\indent%

\index{Rocketship!rent subsidies|(}
The last big source of revenue for Rocketship are the rent subsidies that the State of California offers through SB740. Lines 113–232 of this dissertation's Data Dashboard list 12 years (2011–2022) of subsidy that total over \$43M. In recent years, SB740 adds over \$5M annually to Rocketship's revenue stream. Note that SB740 subsidies continue until either the law is changed or Rocketship goes out of business; subsidies do not stop when a facility is fully paid off.
\index{Rocketship!rent subsidies|(}

\subsection{Debt}%
\label{sec:debt}\indent%

\index{Rocketship!debt|(}
Rocketship has borrowed over 50 times since its founding in 2006\footnote{Full details of Rocketship's borrowings are in this dissertation's Google spreadsheet (see the previous footnote) in the tab \textit{Dashboard} starting on lines 10–63.} and their annual, consolidated financial statements provide debt summaries starting in 2012. The totals are shown in \prettyref{tab:total_debt} on p.\pageref{tab:total_debt} below.

The annual increase (or decrease) in debt year-over-year is shown in \prettyref{tab:total_debt}. One can immediately observe that the changes in debt year-over-year are quite pronounced. The annual changes range from a low of 86\% to a high of 155\% with an average of just over 100\% per year. (An increase of 86\% means that the absolute amount of debt decreased. An increase of 155\% means that debt increased one and a half times the previous year.)

\begin{table}[ht]
  \caption[Total Debt, 2012-2022]{\textit{Total Debt, 2012-2022}}%
  \label{tab:total_debt}
  \begin{tabular}{rrr}
    \toprule
    \textbf{Year} & \textbf{Total Debt} & \multirow[t]{2}{1.1in}{\textbf{Year-Over-Year\\Increase}}\\
    \\
    \midrule
    2012 & \$47,046,048 & \\
    2013 & \$57,078,166 & 121.32\% \\
    2014 & \$88,383,082 & 154.85\% \\
    2015 & \$75,904,098 &  85.88\% \\
    2016 & \$104,857,696 & 138.14\% \\
    2017 & \$136,652,562 & 130.32\% \\
    2018 & \$129,391,897 &  94.69\% \\
    2019 & \$163,598,844 & 126.44\% \\
    2020 & \$168,701,124 & 103.12\% \\ 
    2021 & \$196,416,045 & 116.43\% \\
    2022 & \$186,550,566 &  85.89\% \\
    \bottomrule
  \end{tabular}
\end{table}
\index{Rocketship!debt|)}

\subsection{The New Markets Tax Credit (NMTC) Program}%
\label{sec:NMTC}\indent%

\index{New Markets Tax Credit Program|(}
One form of debt that Rocketship has used numerous times is the New Markets Tax Credit (NMTC) program. This is one of six programs offered by the Community Development Financial Institutions Fund (CDFI Fund) of the U.S. Department of the Treasury. In essence, the NMTC program offers a 39\% tax credit on qualified investments in ``Low-Income Communities''. The credits are spread out over 7 years: 5\% of the amount invested for the first 3 years, and 6\% for the remaining 4 years. These credits can be applied to federal income taxes due from other investments. 

Charter schools operating in economically depressed areas qualify for tax credits. 

It is incentives like these incentives make the NMTC popular, coupled with reduced risk. The tax credit investment is nearly without risk because the tax credit is guaranteed as long as the charter school remains open. If the school stays open for seven years, the risk is nearly zero.

\index{New Markets Tax Credit Program!example|(}
An example will help make it clear how the NMTC works. Suppose we have a high wealth individual with a marginal tax rate of 37\% (the highest bracket), and suppose that this high wealth individual gets a return 10\% per annum on their investments. Further, suppose this individual has \$2,351,000 to invest. The investor could divide that amount into a \$1,000,000 NMTC investment and \$1,351,000 investment in something other than a qualified NMTC investment.

In the NMTC case, every year for the first three years, the investor has \$1,000,000 $\times$ a 10\% return $\times$ 37\% income tax = \$37,000 tax due on \$100,000 profit, The investor gets back their original amount plus profit on the original amount minus taxes on the profit or (\$1,000,000 + (\$1,000,000 × 10\%) - (\$1,000,000 × 0.37) = \$1,063,000 after taxes, in addition to a \$1,000,000 × 5\% = \$50,000 tax credit. The last four years, the tax credit rises to \$60,000.
  
In the non-qualifying investment, the investor has \$1,351,000 $\times$ 10\% return $\times$ 37\% income tax = \$50,000 tax due, which is exactly equal to the tax credit of the NMTC case. The investor also has the return on the investment whose tax due was equal to the NMTC tax credit: \$1,351,000 + (\$1,351,000 $\times$ 10\%) = \$1,486,000. In years 4–7, the investor can invest \$1,622,000 which generates a \$60,000 tax due.

Combining the returns from the NMTC case with those from the non-NMTC case, the investor gets back \$1,063,000 + \$1,486,000 = \$2,549,000 for an investment of \$2,351,000, a 8.47\% net return after taxes. In years 1–3, the after tax return to the high wealth individual is 6.3\%, ((\$2,549,000 $÷$ \$2,351,000) $×$ 100) so the NMTC after tax return is 33.3\% higher than investing \$2,351,000, but at much lower risk. In years 4–7, the net return is slightly less (7.3\%) and the investment needed slightly higher (\$2,622,622).
\index{New Markets Tax Credit Program!example|)}
\index{New Markets Tax Credit Program|)}

\subsection{Rocketship's Worth}%
\label{sec:rocketship-worth}\indent%

\index{Rocketship!net worth|(}
When all is said and done, Rocketship's (including Launchpad Development Company's assets) net assets are how much it is worth. So, it is instructive to look at how Rocketship's worth has varied over the years.
\begin{itemize}
  \item Rocketship's net assets from 2010 to 2022 have always been positive as seen in \prettyref{tab:net_assets_annual_change} on p.\pageref{tab:net_assets_annual_change}, although they have risen in some years and fallen in others, sometimes considerably so.
  \item Rocketship's Compound Annual Growth Rate (CAGR) started with a stratospheric 300+\% and then gradually fell to a merely excellent 25–30\% starting in 2017, a rate of return that would be the envy of any venture fund or private equity firm.
 \item This CAGR has been obtained with very little funding from the founders, and very little might actually be zero. This makes their rate of return on investment theoretically infinite.
\end{itemize}

\begin{table}[ht]
  \caption[Net Assets, 2010–2022]{\textit{Net Assets, 2010–2022}}%
  \label{tab:net_assets_annual_change}
  \begin{tabular}{rrcc}
    \toprule
    \textbf{Year} & \textbf{Net Assets} & \multirow[t]{2}{1.1in}{\textbf{Year-Over-Year\\Increase}} & \textbf{CAGR}\\
    \\
    \midrule
    2010 &   \$2,218,964	&            & \\
    2011 &   \$9,212,140	&   315.16\% & 315.16\% \\
    2012 &  \$11,933,099	&    29.54\% & 103.75\% \\
    2013 &  \$15,881,210	&    33.09\% &  92.71\% \\ 
    2014 &  \$13,356,528	&   -15.90\% &  56.63\% \\
    2015 &  \$10,562,747	&   -20.92\% &  36.62\% \\
    2016 &  \$16,931,464	&    60.29\% &  40.31\% \\
    2017 &  \$17,536,163	&     3.57\% &  34.36\% \\
    2018 &  \$20,883,606	&    19.09\% &  32.36\% \\
    2019 &  \$24,084,572        &    10.12\% &  30.34\% \\
    2020 &  \$24,617,294        &     2.21\% &  27.21\% \\
    2021 &  \$38,231,318	&    55.30\% &  29.54\% \\ 
    2022 &  \$33,442,645        &   -12.53\% &  25.37\% \\
    \bottomrule
  \end{tabular}
\end{table}
\index{Rocketship!net worth|)}

\section{Gaps,  Anomalies, and Discrepancies}%
\label{sec:gaps_anomolies_discrepencies}\indent%

This section is concerned with what wasn't found during this dissertation's investigation. Gaps are where data were expected, but none were found. Anomalies are where data were found, but differed from what were expected, and discrepancies are where data were found but conflicted with other data.

In an enterprise as large as Rocketship is now (with a \$190M+ budget in 2022), there are bound to be unintentional gaps, anomalies, and discrepancies without any implication of nefarious intent. Further, as Berman and Knight emphasize, accounting involves making assumptions, estimates, and judgment calls; it is not an exact science.
\begin{displayquote}%[\parencite[4-5]{Berman.Knight2013}]
The art of accounting and finance is the art of using limited data to come as close as possible to an accurate description of how well a company is performing.
\end{displayquote}

So, the mere existence of gaps, anomalies, and discrepancies is not an indication of fraud. Fraud is deliberate, but gaps, anomalies, and discrepancies occur because of differing assumptions, simple oversight, recording errors, or (unfortunately)  incompetence.

\subsection{Gaps}%
\label{sec:gaps}\indent%

Gaps are where data were expected, but not found. No significant gaps were found.

\subsection{Anomalies}%
\label{sec:anomalies}\indent%

Anomalies are where data were found, but were not what was expected. Several anomalies were found.

\index{Rocketship!anomalies|(}
\begin{itemize}
  \item It appears that the Rocketship Business Committee only reviews and approves already signed checks in excess of \$100,000 rather than reviewing and approving purchase orders (i.e.\ before signature). It is not known if those checks have been sent to their respective payees.\index{Rocketship!anomalies!check review}

  \item The audited financial statements use a level of materiality (\$300K) that is three times higher than that used by a public school district (LASD) whose budget is half the size of Rocketship's, i.e. Rocketship's level of materiality is 50\% higher than expected.\index{Rocketship!anomalies!level of materiality}

  \item Administrative expenses, compared to total expenses, seem unusually high. Using functional expenses from the 2021-2022 school year as an example, Rocketship spent \$151,416,849 on educational programs, \$33,683,700 on program support, and \$46,401,574 on management and general expenses.  The management and general expenses are thus approximately 30\% of what was spent on educational programs. In a Business Committee presentation, \textcite[28]{Mukhopadhyay2013}, Rocketship says that the fees they charge individual schools are 35\% of revenue, consisting of a 15\% management fee and a 20\% facility fee, so 30\% is in line with what the Business Committee expects.\index{Rocketship!anomalies!administrative expenses}

  \item The functional expenses that Rocketship has chosen to use in their financial statements differ from the list used in IRS Form 990, Part IX\@. This makes it nearly impossible to cross check (triangulate) data from Form 900 and the audited annual financial statements.\index{Rocketship!anomalies!differing functional expenses}

  \item  For the years ended 2019–2022, accounting expenses were \$166,059 in 2019, more than doubling to \$423,683 in 2020, roughly halving to \$264,784 in 2021, before more than tripling to \$848,221 in 2022. No mention is made of these substantial swings in accounting expenses in the \textit{Notes to Consolidated Financial Statements} for 2022.\index{Rocketship!anomalies!large swings in accounting expenses}

  \item In 2022, a total of \$2,635,011 was spent on travel which is over \$200,000 per month. This represents perhaps 50 cross-country business class flights per month (\$1500 flight with a five day stay at a five-star hotel at \$500/night).  More modest flights and hotel (\$500 + 5×\$300) allow 100 trips per month.  No explanation either for the need for this much travel or nor its cost was provided in the \textit{Notes to Consolidated Financial Statements}, especially in this day and age of Zoom. An argument could be made that swings this large are material and so must be explained.\index{Rocketship!anomalies!travel expenses}
  
  \item In general, Rocketship spents more per student than their comparable home public school district in Santa Clara County does. This is unusual because, again in general, Rocketship schools have a higher student-to-teacher ratio than surrounding districts. Unexplained is why Rocketship has a higher cost per student. Charter schools were supposed to be more efficient than public schools because they did not have to adhere to bothersome and costly regulations that public schools did.\index{Rocketship!anomalies!per pupil expenditure}
\end{itemize}
\index{Rocketship!anomalies|)}

\subsection{Discrepancies}%
\label{sec:discrepancies}\indent%
\index{Rocketship!discrepancies|(}

Discrepancies are where two or more sources of the same data differ when they ought to agree.

The following discrepancies were found:
\begin{itemize}
  \item Annual Financial Statements and Form 990s\\
  The annual audited financial statements have several entries which also appear in the IRS Form 990, the federal tax return for organizations exempt from income tax, i.e.\ charities, religious organizations, private foundations, some political organizations, and other non-profits. For example, on June 30, 2015, the Consolidated Statement of Financial Position for 2014-2015 shows net assets to be \$10,562,747 (p.3) whereas the Form 990 (2014) show them to be \$13,968,882, a 32\% difference. Analysis of this discrepancy is limited and is insufficient to determine if the difference is the result of differing accounting practices or is a reflection of a more serious underlying problem.\index{Rocketship!discrepancies!Form 990 vs financial statements}

  Similar discrepancies also exist for functional expenses, among other categories.
 \item For year 2018–2019, salaries are shown as \$54,294,263 on the audited statement of functional expenses. Yet, adding lines 5 (executive compensation), 7 (other salaries), 8 (pension plan), and 9 (other employee benefits) from Form 990 (2018–2019) yields \$54,516,782 which is close, but not quite the same as the amount shown in the audited statements. Further, it is not even clear that those lines and only those should sum to the same amount as ``Salaries'' in the audited statement of functional expenses.  For example, should pensions be counted as part of salaries? \index{Rocketship!discrepancies!Form 990 vs audited salaries}

  \item The total contributions (i.e.\ donations, grants) for all Rocketship schools in 2021-2022 is listed as \$7,075,182. The sum total of Object Codes 8980-8999 (Contributions) for all ten Santa Clara County Rocketship Schools is \$3,326,893. These two numbers are clearly not the same, but neither are the schools covered. The first consists of all 23 Rocketship schools in the U.S.; the latter consist only of ten schools in Santa Clara County. Considerable work beyond the scope of this investigation would be needed to determine if all the reported contributions agree.\index{Rocketship!discrepancies!audited contributions vs object code sums}

  The SACS Object Codes 8980-8999 are where contributions are recorded in Rocketship's unaudited actuals (and reported to the California Department of Education). The Department of Education makes available a Microsoft Access database with data for specific object codes or groups of object codes for every charter school in California. Summing each school's Object Code 8980-8999 for a test year (YE2020) does not agree with what's is reported on line 11 of Rocketship's Consolidated Statement of Activities for that year, nor does it agree with what was reported on Rocketship's IRS Form 990 that year.\index{Rocketship!discrepancies!object code sums vs financial statements}

  Several questions remain: Are the differences merely differences in accounting standards of the California Department of Education and the IRS\@? Or, are the differences choices that Rocketship has made? And if so, why? One final question: the entries for Object Codes 8980-8999 are the only entries which have the form of a positive number under ``restricted'' and an identical number, but negative, under ``unrestricted''. The total is naturally zero, which begs the question of what exactly is being reported.\index{Rocketship!discrepancies!unclear reporting}

\end{itemize}
\index{Rocketship!discrepancies|)}

\section{Issues of Equality and Equity}%
\label{sec:issues_equality_equity}\indent%

\index{Rocketship!vision and goals|(}
Ostensibly, equality and equity are at the heart of why Rocketship exists. Their vision is to ``eliminate the achievement gap in our lifetime'' \parencite{RSE2017}.\footnote{Uncharitably, depending on whose lifetime Rocketship is referring to, the elimination of the achievement gap could be 30–60 years out. Included in this span of years is at least one pandemic, one major earthquake, several depressions or recessions, several wars, and numerous government shutdowns.} Their mission is to ``catalyze transformative change in low-income communities through a scalable and sustainable public school model that propels student achievement, develops exceptional educators, and partners with parents who enable high-quality public schools to thrive in their community.'' (ibid.) These are laudable goals, but not unique to Rocketship. Other charter schools, and even public schools have similar goals.
\index{Rocketship!vision and goals|)}

\index{Rocketship!per-pupil spending|(}
At a high level, Rocketship spent in 2021-22 (based on the \textit{Santa Clara County 2022-2023 Annual Charter Schools Data Book} which is the latest available) on its eight Santa Clara County authorized schools between \$16,128 and \$23,995 per pupil which is from \$2,090 less than that school's comparable district to \$5,777 more. Six out of eight schools spent more per student than their comparable district does \parencite{SCCOE2014}.
\index{Rocketship!per-pupil spending|)}

\index{Rocketship!claims of academic performance|(}
Rocketship locates all of its schools in high poverty areas\footnote{High poverty areas are defined as areas where 20\% live in poverty or where the median family income less than 80\% of the area median family income\parencite[13-14]{CDFI2020}} where chronically underfunded public schools struggle to provide a quality education to all. Had they not done so, investors would not have been able to take advantage of the NMTC program. Despite being located in high poverty areas, Rocketship claims that its elementary schools are among the best in the nation \parencite{Abousalem2021} because their API or CAASPP test scores are higher than their surrounding district and higher than the California average. However, an argument can be made that all Rocketship can actually claim is that their students are among the best standardized test takers in the nation; there is no evidence that Rocketship students who continue their education (middle school, high school, college, university) do any better than public schools students. As previously mentioned, \textcite{Lubienski.Lubienski2014} have shown that the NAEP test results of public schools are higher than those of charter schools, all things considered. Of course this does not mean that Rocketship couldn't be an outlier whose students do better in the long run than those of other public or charter schools, but the only evidence that has been presented is \textcite{Raymond.etal2023}. Like other CREDO publications, the findings of \textcite{Raymond.etal2023} have been disputed.\footnote{Stanford University's CREDO (Center for Research on Education Outcomes) makes the case that ``from 2015 to 2019, the typical charter school student in our national sample had reading and math gains that outpaced their peers in the traditional public schools (TPS) they otherwise would have attended''. Reviewing the lastest CREDO report for the NEPC, Joseph Ferrare said, ``[This] CREDO report compares charter school students’ learning in reading and math to students in traditional public schools. The report should be approached with caution by policymakers given the nonexperimental design that renders it unable to fully account for the factors that drive families to choose charter schools. In addition, the report presents its findings using an unconventional metric that makes it difficult to understand the policy implications, potentially misleading policymakers. The magnitude of the main findings fails to meet the minimum threshold experts consider to be a meaningful educational intervention.'' \parencite{Ferrare2023}}

The evidence that children learn more when in front of computers is decidedly mixed. Some meta-analyses report positive outcomes, but small effect sizes. It has, however, been known for years that virtual charter schools (100\% computerized) almost universally fail to perform better than schools using a blended approach or those with 100\% teacher-led instruction. These issue were discussed in \prettyref{sec:types-instruction} on p.\pageref{sec:types-instruction}.
\index{Rocketship!claims of academic performance|)}


\index{Rocketship!teacher|(}
In 2023, Rocketship teachers, on average, have barely more than three years of teaching experience \parencite{SCCOE2014}. In comparison, San José Unified School District (SJUSD) has 81.1\% of its teachers with more than three years of experience \parencite{USNews2023}.\index{Rocketship!teacher!experience} Further, no Rocketship school authorized by the Santa Clara County Board of Education (SCCBOE) has an average teacher salary that exceeds \$70,000, whereas SJUSD's average salary is just shy of \$90,000. SCCBOE-authorized schools have on average a 30\% annual teacher attrition rate \parencite{SCCOE2014}.\index{Rocketship!teacher!salaries} In 2014–2015 and 2015-2016, Rocketship reported in a bond prospectus that roughly one-third of its teachers didn't return the following years \parencite{CSFA2017}.\index{Rocketship!teacher!turnover} Given these figures, it would indeed be extraordinary if Rocketship schools did substantially better than their home district. A more likely reason for their (initial) successes is that the early cohorts at a school have the most engaged and supportive parents who have high expectations for their children's academic achievement. One would expect that over time that both scores and enrollment would regress to the mean, and that's exactly what happened \parencite{SCCOE2014}. 
\index{Rocketship!teachers|)}

\index{Rocketship!annual academic growth|(}
Rocketship, whose explicit goal is to close the achievement gap, claims on their web site, that their students ``\ldots{} receive 5-7 Additional \textit{Months} of Learning'' [emphasis added] \parencite{RSED2023}, but this is not what is reported in \textcite{Raymond.etal2023}. \textit{Appendix A: Average Annual Academic Growth of CMOs and Networks, Reading and Math} \textcite[132]{Raymond.etal2023} shows Rocketship's estimated annual growth in reading and math to be 0.166 and 0.239 respectively. For a standard 180 day year (in California), this is just under 30 days growth in reading and just over 43 days in math, neither are anywhere near the 5 to 7 months as claimed.

The CREDO report also claims that Rocketship is a ``Gap Buster'' CMO, where ``Gap Buster'' is defined as ``moved their achievement ahead of their respective state’s average performance.'' \parencite[14]{Raymond.etal2023}. The report claims that ``\ldots{} more than 1,000 schools [in the U.S.] have eliminated learning disparities for their students'' \parencite[14]{Raymond.etal2023} which may or may not be true because the comparison is not an all-else-equal comparison, i.e.\ it is not commensurable. As \textcite[16]{Ashworth.etal2021} say, ``Failure to respect all-else-equal conditions is the most important source of incommensurability.'' The claim is only true if the demographics of a charter school and its home district are the same, and there are valid reasons to believe they aren't (citation)
\index{Rocketship!annual academic growth|(}

%%% Local Variables:
%%% mode: latex
%%% TeX-master: "Rocketship_Education-An_Exploratory_Public_Policy_Case_Study"
%%% End:
